\documentclass[12pt]{article}
\usepackage{pmmeta}
\pmcanonicalname{LogarithmSeries}
\pmcreated{2013-03-22 18:56:13}
\pmmodified{2013-03-22 18:56:13}
\pmowner{pahio}{2872}
\pmmodifier{pahio}{2872}
\pmtitle{logarithm series}
\pmrecord{8}{41791}
\pmprivacy{1}
\pmauthor{pahio}{2872}
\pmtype{Topic}
\pmcomment{trigger rebuild}
\pmclassification{msc}{33B10}
\pmrelated{TaylorSeriesOfArcusSine}
\pmrelated{TaylorSeriesOfArcusTangent}

\endmetadata

% this is the default PlanetMath preamble.  as your knowledge
% of TeX increases, you will probably want to edit this, but
% it should be fine as is for beginners.

% almost certainly you want these
\usepackage{amssymb}
\usepackage{amsmath}
\usepackage{amsfonts}

% used for TeXing text within eps files
%\usepackage{psfrag}
% need this for including graphics (\includegraphics)
%\usepackage{graphicx}
% for neatly defining theorems and propositions
 \usepackage{amsthm}
% making logically defined graphics
%%%\usepackage{xypic}

% there are many more packages, add them here as you need them

% define commands here

\theoremstyle{definition}
\newtheorem*{thmplain}{Theorem}

\begin{document}
The derivative of\, $\ln(1\!+\!x)$\, is $\displaystyle\frac{1}{1\!+\!x}$, which can be represented as the sum of geometric series:
$$\frac{1}{1\!+\!x} \;=\; 1-x+x^2-x^3+-\ldots \qquad\mbox{for}\;\; -1 < x < 1.$$
Integrating both \PMlinkescapetext{sides} from 0 to $x$ gives
\begin{align}
\ln(1\!+\!x) \;=\; x-\frac{x^2}{2}+\frac{x^3}{3}-\frac{x^4}{4}+-\ldots \qquad\mbox{for}\;\; -1 < x < 1.
\end{align}
which is valid on the whole open interval of convergence \,$-1 < x < 1$\, of this power series and in 
\PMlinkescapetext{addition} for\, $x = 1$, as one may prove.

Replacing $x$ with $-x$ in (1) yields the series
\begin{align}
\ln(1\!-\!x) \;=\; -x-\frac{x^2}{2}-\frac{x^3}{3}-\frac{x^4}{4}-\ldots \qquad\mbox{for}\;\; -1 < x < 1.
\end{align}
Subtracting (2) from (1) gives
\begin{align}
\ln\frac{1\!+\!x}{1\!-\!x} \;=\; 2\left(x+\frac{x^3}{3}+\frac{x^5}{5}+\frac{x^7}{7}+\ldots\right)
\end{align}
which also is true for\, $-1 < x < 1$.\, Here the inner function of the logarithm attains all positive real values when\, $0 < x < 1$ (its \PMlinkname{graph}{Graph2} is a \PMlinkname{hyperbola}{Hyperbola2} with \PMlinkname{asymptotes}{AsymptotesOfGraphOfRationalFunction} \,$x = 1$\, and\, $y = -1$).\, Thus, in principle, the series (3) can be used for calculating any values of \PMlinkname{natural logarithm}{NaturalLogarithm2}.\, For this purpose, one could denote
$$\frac{1\!+\!x}{1\!-\!x} \;:=\; t,$$
which implies
$$x \;=\;\frac{t\!-\!1}{t\!+\!1},$$
and accordingly
\begin{align}
\ln{t} \;=\; 2\left[\frac{t\!-\!1}{t\!+\!1}
            +\frac{1}{3}\left(\frac{t\!-\!1}{t\!+\!1}\right)^3
            +\frac{1}{5}\left(\frac{t\!-\!1}{t\!+\!1}\right)^5+\ldots\right]\!.
\end{align}
For example, 
$$\ln{3} \;=\; 2\left(\frac{1}{2}+\frac{1}{3\cdot2^3}+\frac{1}{5\cdot2^5}+\ldots\right)\!.$$
The convergence of (4) is the slower the greater is $t$.



%%%%%
%%%%%
\end{document}
