\documentclass[12pt]{article}
\usepackage{pmmeta}
\pmcanonicalname{IndexOfSpecialFunctions}
\pmcreated{2013-03-22 14:40:06}
\pmmodified{2013-03-22 14:40:06}
\pmowner{rspuzio}{6075}
\pmmodifier{rspuzio}{6075}
\pmtitle{index of special functions}
\pmrecord{35}{36269}
\pmprivacy{1}
\pmauthor{rspuzio}{6075}
\pmtype{Topic}
\pmcomment{trigger rebuild}
\pmclassification{msc}{33-00}
\pmrelated{ComplexFunction}
\pmrelated{ExponentialIntegral}
\pmrelated{SpecialCasesOfHypergeometricFunction}
\pmrelated{PropertiesOfOrthogonalPolynomials}

% this is the default PlanetMath preamble.  as your knowledge
% of TeX increases, you will probably want to edit this, but
% it should be fine as is for beginners.

% almost certainly you want these
\usepackage{amssymb}
\usepackage{amsmath}
\usepackage{amsfonts}

% used for TeXing text within eps files
%\usepackage{psfrag}
% need this for including graphics (\includegraphics)
%\usepackage{graphicx}
% for neatly defining theorems and propositions
%\usepackage{amsthm}
% making logically defined graphics
%%%\usepackage{xypic}

% there are many more packages, add them here as you need them

% define commands here
\begin{document}
The term \PMlinkescapetext{``special function''} is not a completely precise mathematical term.  It usually refers to a function of one or more real or complex variables which is either of use in some application or interesting in its own right, and hence has been studied enough to warrant giving it a name.  Special functions are usually named after the mathematician who first introduced them or contributed much to their theory although, as in the rest of mathematics, such attributions are not always accurate, and they should be taken with a grain of salt.

\subsection{\PMlinkid{Elementary Functions}{6420}}
\begin{itemize}
\item exponential
\item logarithm
\item \PMlinkid{trigonometric functions}{4676}
\item \PMlinkid{cyclometric functions}{6169}
\item hyperbolic functions
\item area functions
\item Gudermannian function
\item \PMlinkid{$\operatorname{sinc}$ function}{5744}

\end{itemize}

\subsection{Antiderivatives of elementary functions}
\begin{itemize}
\item error function
\item logarithmic integral
\item exponential integral
\item cosine integral
\item sine integral
\item hyperbolic sine integral
\item Fresnel integrals
\item elliptic integrals
\end{itemize}

\subsection{Gamma and related functions}
\begin{itemize}
\item gamma function
\item beta function
\item polygamma functions
\item Barnes function
\end{itemize}

\subsection{Functions defined as solutions of linear differential equations}
\begin{itemize}
\item {\bf Bessel functions}
\item hypergeometric function
\item confluent hypergeometric function
\item {\bf generalized hypergeometric functions}
\item Hermite polynomials
\item Legendre polynomials
\item associated Laguerre polynomials
\item spherical harmonics
\item surface harmonics
\item Lam\'e function
\item Heun function
\end{itemize}

\subsection{Functions defined as solutions of non-linear equations}
\begin{itemize}
\item Painlev\'e transcendents
\item Emden function
\item Lambert W function 
\item Airy functions
\end{itemize}

\subsection{Abelian functions}
\begin{itemize}
\item Jacobi theta functions
\item Weierstrass sigma function
\item Jacobi Zeta function
\item Weierstrass zeta function
\end{itemize}

\subsection{Zeta Functions}
\begin{itemize}
\item general Zeta functions (in the sense of Jorgensen and Lang)
\item Hecke zeta function
\item Hurwitz zeta function
\item $L$-functions
\item Riemann zeta function
\item Zeta functions of surfaces
\item Zeta functions of graphs
\item Zeta functions of operators
\end{itemize}
%%%%%
%%%%%
\end{document}
