\documentclass[12pt]{article}
\usepackage{pmmeta}
\pmcanonicalname{ErrorFunction}
\pmcreated{2013-03-22 14:46:51}
\pmmodified{2013-03-22 14:46:51}
\pmowner{rspuzio}{6075}
\pmmodifier{rspuzio}{6075}
\pmtitle{error function}
\pmrecord{10}{36429}
\pmprivacy{1}
\pmauthor{rspuzio}{6075}
\pmtype{Definition}
\pmcomment{trigger rebuild}
\pmclassification{msc}{33B20}
\pmrelated{AreaUnderGaussianCurve}
\pmrelated{ListOfImproperIntegrals}
\pmrelated{UsingConvolutionToFindLaplaceTransform}
\pmdefines{complementary error function}

\endmetadata

% this is the default PlanetMath preamble.  as your knowledge
% of TeX increases, you will probably want to edit this, but
% it should be fine as is for beginners.

% almost certainly you want these
\usepackage{amssymb}
\usepackage{amsmath}
\usepackage{amsfonts}

% used for TeXing text within eps files
%\usepackage{psfrag}
% need this for including graphics (\includegraphics)
%\usepackage{graphicx}
% for neatly defining theorems and propositions
%\usepackage{amsthm}
% making logically defined graphics
%%%\usepackage{xypic}

% there are many more packages, add them here as you need them

% define commands here
\begin{document}
The \emph{error function} ${\rm erf} \colon \mathbb{C} \to \mathbb{C}$ is defined as follows:
 $${\rm erf} (z) = {2 \over \sqrt{\pi}} \int_0^z e^{-t^2} \, dt$$
The \emph{complementary error function} ${\rm erfc} \colon \mathbb{C} \to \mathbb{C}$ is defined as
 $${\rm erfc} (z) = {2 \over \sqrt{\pi}} \int_z^\infty e^{-t^2} \, dt$$

The name ``error function'' comes from the role that these functions play in the theory of the normal random variable.  It is also worth noting that the error function is a special case of the confluent hypergeometric functions and of the Mittag-Leffler function.

\textbf{Note.}\,  By \PMlinkname{Cauchy integral theorem}{SecondFormOfCauchyIntegralTheorem}, the choice path of integration in the definition of ${\rm erf}$ is irrelevant since the integrand is an entire function.   In the definition of ${\rm erfc}$, the path may be taken to be a half-line parallel to the positive real axis with endpoint $z$.

%%%%%
%%%%%
\end{document}
