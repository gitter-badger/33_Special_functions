\documentclass[12pt]{article}
\usepackage{pmmeta}
\pmcanonicalname{DerivationOfIntegralRepresentationsOfJacobivarthetaFunctions}
\pmcreated{2013-03-22 14:39:54}
\pmmodified{2013-03-22 14:39:54}
\pmowner{rspuzio}{6075}
\pmmodifier{rspuzio}{6075}
\pmtitle{derivation of integral representations of Jacobi $\vartheta$ functions}
\pmrecord{26}{36263}
\pmprivacy{1}
\pmauthor{rspuzio}{6075}
\pmtype{Derivation}
\pmcomment{trigger rebuild}
\pmclassification{msc}{33E05}

\endmetadata

% this is the default PlanetMath preamble.  as your knowledge
% of TeX increases, you will probably want to edit this, but
% it should be fine as is for beginners.

% almost certainly you want these
\usepackage{amssymb}
\usepackage{amsmath}
\usepackage{amsfonts}

% used for TeXing text within eps files
%\usepackage{psfrag}
% need this for including graphics (\includegraphics)
%\usepackage{graphicx}
% for neatly defining theorems and propositions
%\usepackage{amsthm}
% making logically defined graphics
%%%\usepackage{xypic}

% there are many more packages, add them here as you need them

% define commands here
\begin{document}
By rearranging the Fourier series of $\cos (ux)$, one obtains the series
 $${ \pi \cos (u x) \over 2 u \sin (\pi u)} = {1 \over 2 u^2} + \sum_{n=1}^\infty (-1)^n {\cos (nx) \over u^2 - n^2}$$
This equation which is valid for all real values of $x$ such that $-\pi \le x \le \pi$ and all non-integral complex values of $u$.  By comparison with the convergent series $\sum_{n=0}^\infty 1/n^2$, it follows that this series is absolutely convergent.  Note that this series may be viewed as a Mittag-Leffler partial fraction expansion.  

Let $y$ be a positive real number.  Multiply both \PMlinkescapetext{sides} by $2 u e^{-y u^2}$ and integrate.
 $$\int_{i-\infty}^{i + \infty} {\pi \cos (u x) e^{-yu^2} \over \sin (\pi u)} \, dv = 2 \int_{i-\infty}^{i + \infty} e^{-y u^2} \left[ {1 \over 2 u^2} + \sum_{n=0}^\infty (-1)^n {\cos (nx) \over u^2 - n^2} \right] \, u \, du$$

Because of the exponential, the integrand decays rapidly as $u \to i \pm \infty$ provided that $\Re u > 0$, and hence the integral converges absolutely.  Make a change of variables $v = u^2$
 $$= \int_P e^{-y v} \left[ {1 \over 2 v} + \sum_{n=1}^\infty (-1)^n {\cos(nx)\over v - n^2} \right] \, dv$$
The contour of integration $P$ is a parabola in the complex $v$-plane, symmetric about the real axis with vertex at $v = -1$, which encloses the real axis.  Its equation is $\Re v + 1 = 2 (\Im v)^2$

Let $S_m$ ($m$ is an integer) be the straight line segment joining the points $v = (i + m + 1/2)^2$ and $v = (i - m - 1/2)^2$.  Along this line segment, we may bound the integrand in absolute value as follows:
 $$\left| \sum_{n=1}^\infty (-1)^n {\cos (nx) \over v - n^2} \right| \le \sum_{n=1}^\infty {(-1)^n \over |v - n^2|} \le \sum_{n=1}^\infty {(-1)^n \over |v_m - n^2|}$$
where $v_m = m^2 + m - 3/4$ is the point of intersection of $S_m$ with the real axis.  To proceed further, we break up the last summation into two parts.

Since the squares closest in absolute value to $v_m$ are $m^2$ and $(m+1)^2 = m^2 + 2m + 1$, it follows that $|v_m - n^2| \ge |m - 3/4|$ for all $m,n$.  Hence, we have
 $$\sum_{i=1}^{2m} {1 \over |v_m - n^2|} \le {2m \over m - 3/4} \le 8$$

When $n > 2m$, we have $n^2 \ge (2m+1)^2 = 4m^2 + 4m + 1 > 4m^2 + 4m - 3 = 4 v_m$.  Hence, $|n^2 - v_m| > 3 n^2 /4$ and
 $$\sum_{n = 2m+1}^\infty {1 \over |v_m - n^2|} < {4 \over 3} \sum_{n = 2m+1}^\infty {1 \over n^2} < {4 \over 3} \sum_{n = 1}^\infty {1 \over n^2} = {2 \pi \over 9}$$

Finally $1/(2 v_m) < 1/2$ since $v_m > 1$ when $m \ge 1$.  Also, $|e^{-y v}| = e^{-y \Re v} - e^{-y v_m} < e^{-y m^2}$.  From these observations, we conclude that
 $$\left| \int_{S_m} e^{-y v} \left[ {1 \over 2 v} + \sum_{n=1}^\infty (-1)^n {\cos(nx)\over v - n^2} \right] \, dv \right| < e^{-y m^2} \left( 1 + 8 + {2 \pi \over 9} \right) \int_{S_m} dv = (4m + 2) \left( 9 + {2 \pi \over 9} \right) e^{-y m^2}$$
Note that this quantity approaches 0 in the limit $m \to \infty$.

Let $P_m$ be the arc of the parabola $P$ bounded by the endpoints of $S_m$.  Together, $S_m$ and $P_m$ form a closed contour which encloses poles of the integrand.  Hence, by the residue theorem , we have 
 $$\int_{P_m} e^{-y v} \left[ {1 \over 2 v} + \sum_{n=1}^\infty (-1)^n {\cos(nx)\over v - n^2} \right] \, dv + \int_{S_m} e^{-y v} \left[ {1 \over 2 v} + \sum_{n=1}^\infty (-1)^n {\cos(nx)\over v - n^2} \right] \, dv =$$
 $$2 \pi i \sum_{n = 1}^m (-1)^n \cos(nx) e^{-n^2 y}$$

Taking the limit $m \to \infty$ we obtain
 $$\int_{P} e^{-y v} \left[ {1 \over 2 v} + \sum_{n=1}^\infty (-1)^n {\cos(nx)\over v - n^2} \right] \, dv = 2 \pi i \left( {1 \over 2} + \sum_{n = 1}^\infty (-1)^n \cos(nx) e^{-n^2 y} \right)$$
Going back to the beginning of the proof, where the integral on the left hand side was expressed as an integral with respect to $u$, we obtain
 $$\int_{i-\infty}^{i + \infty} {\pi \cos (u x) e^{-yu^2} \over \sin (\pi u)} \, dv = 2 \pi i \left( {1 \over 2} + \sum_{n = 1}^\infty (-1)^n \cos(nx) e^{-n^2 y} \right)$$
 Making a change of variables $x = 2z, y = -i \pi \tau$ and tidying up some, we obtain
 $$\int_{i-\infty}^{i + \infty} {\cos (2 u z) e^{i \pi \tau u^2} \over \sin (\pi u)} \, dv = i \left( 1 + 2 \sum_{n = 1}^\infty (-1)^n e^{i \pi n^2 \tau} \cos(2 n z) \right) = i \vartheta_4 (z | \tau)$$

Because of the initial assumption about the Fourier series, we only know that this formula is valid when $\tau$ is purely imaginary with strictly positive imaginary part and $z$ is real and $\pi/2 < z < \pi/2$.  However, we can use analytic continuation to extend the domain of its validity.  On the one hand, the theta function on the right-hand side is analytic for all $z$ and all $\tau$ such that $\Im \tau > 0$.

On the other hand, I claim that the integral on the left hand side is also an  analytic function of $z$ and $\tau$ whenever $\Im \tau > 0$.  To validate this claim, we need to examine the behaviour of the integrand as $u \to i \pm \infty$.  The contribution of the denominator is bounded;
 $$\left| {1 \over \sin \pi u} \right| < c$$
for some constant $c$ whenever $\Im u = 1$.  The absolute value of the cosine in the numerator is easy to bound:
 $$|\cos (2 u z)| \le e^{2 |u| \, |z|}$$
To bound the remaining term, let us examine the argument of the exponential carefully:
 $$\Im (\tau u^2) = 2 \Re \tau \, \Re u + \Im \tau (\Re u)^2 - \Im \tau = \Im \tau \left( \left( \Re u + {\Re \tau \over \Im \tau} \right)^2 - 1 - \left( {\Re \tau \over \Im \tau} \right)^2 \right)$$
Therefore, if $|\Re u| > 1 + 3 |\Re \tau|/(\Im \tau)$, it will be the case that $\Im (\tau u^2) \ge \Im \tau \, (\Re u)^2 / 9$, and so
 $$\left| e^{i \pi \tau u^2} \right| = e^{-\pi \Im (\tau u^2)} \le e^{-\pi \Im \tau \, (\Re u)^2 / 9}$$

Taken together, the estimates of the last paragraph imply that
 $$\left| \int_{i + R}^{i + \infty} {\cos (2 u z) e^{i \pi \tau u^2} \over \sin (\pi u)} \right| < c \int_{i + R}^{i + \infty} e^{2 |u| |z| -\pi \Im \tau \, (\Re u)^2 / 9}$$
when $R > 1 + 3 |\Re \tau|/(\Im \tau)$.  If we impose the further conditions
 $$R > {180 |z| \over \pi \, \Im \tau} \qquad R^2 > {180 |z| \over \pi \, \Im \tau} \qquad,$$ 
it will be the case that
 $$2 |u| |z| - \pi \Im \tau \, (\Re u)^2 / 9 < 2 \Re u \, |z|  + 2 |z| - \pi \Im \tau \, (\Re u)^2 / 9 < $$ 
$$\left( 2 \Re u \, |z| - \pi \Im \tau \, (\Re u)^2 / 180 \right) + \left( 2 |z| - \pi \Im \tau \, (\Re u)^2 / 180 \right) - \pi \Im \tau \, (\Re u)^2 / 10 <$$ 
 $$- \pi \Im \tau \, (\Re u)^2 / 10 \qquad,$$ 
and hence 
 $$\left| \int_{i + R}^{i + \infty} {\cos (2 u z) e^{i \pi \tau u^2} \over \sin (\pi u)} \, du \right| < c \int_{i + R}^{i + \infty} e^{-\pi \Im \tau \, (\Re u)^2 / 10} \, du < {5 c \over \pi \, \Im \tau} R e^{-\pi \Im \tau \, R^2 / 10} \qquad.$$
Likewise, under the same restriction on $R$,
 $$\left| \int_{i - \infty}^{i - R} {\cos (2 u z) e^{i \pi \tau u^2} \over \sin (\pi u)} \, du \right| < c \int_{i + R}^{i + \infty} e^{-\pi \Im \tau \, (\Re u)^2 / 10} \, du < {5 c \over \pi \, \Im \tau} R e^{-\pi \Im \tau \, R^2 / 10} \qquad.$$

Since the contour of integration is compact and the integrand is analytic in a neighborhood of the contour,
 $$\int_{i - R}^{i + R} {\cos (2 u z) e^{i \pi \tau u^2} \over \sin (\pi u)} \, du$$
will be an analytic function of $z$ and $\tau$.  Suppose that $z$ and $\tau$ are restricted to bounded regions of the complex plane and that, furthermore, $Im \tau$ is positive and bounded away from zero.  Then the inequalities of the last paragraph imply that the integral converges uniformly as $R \to \infty$, and hence
 $$\int_{i - \infty}^{i + \infty} {\cos (2 u z) e^{i \pi \tau u^2} \over \sin (\pi u)} \, du$$ is an analytic function of $u$ and $z$ in the domain $\Im \tau > 0$.

Thus, by the fundamental theorem of analytic continuation, we may conclude that
 $$\int_{i-\infty}^{i + \infty} {\cos (2 u z) e^{i \pi \tau u^2} \over \sin (\pi u)} \, dv = i \left( 1 + 2 \sum_{n = 1}^\infty (-1)^n e^{i \pi n^2 \tau} \cos(2 n z) \right) = i \vartheta_4 (z | \tau)$$
throughout this domain.

Finally, integral representations of the remaining three theta functions may be easily obtained from this one by adding the appropriate half-quasiperiods to $z$.
%%%%%
%%%%%
\end{document}
