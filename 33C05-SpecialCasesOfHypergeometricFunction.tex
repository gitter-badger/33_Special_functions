\documentclass[12pt]{article}
\usepackage{pmmeta}
\pmcanonicalname{SpecialCasesOfHypergeometricFunction}
\pmcreated{2013-03-22 18:54:39}
\pmmodified{2013-03-22 18:54:39}
\pmowner{pahio}{2872}
\pmmodifier{pahio}{2872}
\pmtitle{special cases of hypergeometric function}
\pmrecord{8}{41760}
\pmprivacy{1}
\pmauthor{pahio}{2872}
\pmtype{Example}
\pmcomment{trigger rebuild}
\pmclassification{msc}{33C05}
\pmrelated{FrobeniusMethod}
\pmrelated{IndexOfSpecialFunctions}
\pmrelated{GettingTaylorSeriesFromDifferentialEquation}

\endmetadata

% this is the default PlanetMath preamble.  as your knowledge
% of TeX increases, you will probably want to edit this, but
% it should be fine as is for beginners.

% almost certainly you want these
\usepackage{amssymb}
\usepackage{amsmath}
\usepackage{amsfonts}

% used for TeXing text within eps files
%\usepackage{psfrag}
% need this for including graphics (\includegraphics)
%\usepackage{graphicx}
% for neatly defining theorems and propositions
 \usepackage{amsthm}
% making logically defined graphics
%%%\usepackage{xypic}

% there are many more packages, add them here as you need them

% define commands here

\theoremstyle{definition}
\newtheorem*{thmplain}{Theorem}

\begin{document}
Many \PMlinkname{elementary}{ElementaryFunction} and non-elementary transcendental functions may be expressed as special cases of the hypergeometric functions
$$F(a,\,b,\,c;\,x) \;=\; 1+\frac{ab}{1!c}x+\frac{a(a+1)b(b+1)}{2!c(c+1)}x^2
+\frac{a(a+1)(a+2)b(b+1)(b+2)}{3!c(c+1)(c+2)}x^3+\ldots,$$
which are solutions of the hypergeometric equation
$$x(x-1)\frac{d^2y}{dx^2}+(c-(a+b+1))\frac{dy}{dx}-aby \;=\;0.$$
For example:
\begin{itemize}
\item $(1\!+\!x)^n \;=\; F(-n,\,1,\,1;\,-x)$
\item $\ln(1\!+\!x) \;=\; xF(1,\,1,\,2;\,-x)$
\item $\ln\frac{1+x}{1-x} \;=\; 2xF(\frac{1}{2},\,1,\,\frac{3}{2};\,x^2)$
\item $\arcsin{x} \;=\; xF(\frac{1}{2},\,\frac{1}{2},\,\frac{3}{2};\,x^2)$
\item $\arctan{x} \;=\; xF(\frac{1}{2},\,1,\,\frac{3}{2};\,-x^2)$
\item $\sin(m\arcsin{x}) \;=\; mxF(\frac{1+m}{2},\,\frac{1-m}{2},\,\frac{3}{2};\,x^2)$
\item $\cos(m\arcsin{x}) \;=\; F(\frac{m}{2},\,-\frac{m}{2},\,\frac{1}{2};\,x^2)$
\item $T_n(x) \;=\; F(n,\,-n,\,\frac{1}{2};\,\frac{1-x}{2})$\, (Chebyshev polynomials)
\item $P_n(x) \;=\; F(-n,\,n+1,\,1;\,\frac{1-x}{2})$\, (Legendre polynomials)
\item $\displaystyle\int_0^{\frac{\pi}{2}}\!\frac{d\varphi}{\sqrt{1\!-\!x^2\sin^2\varphi}}$ 
$\;=\; \frac{\pi}{2}F(\frac{1}{2},\,\frac{1}{2},\,1;\,x^2)$\, (complete elliptic integral of 1st kind)
\item $\displaystyle\int_0^{\frac{\pi}{2}}\!\sqrt{1\!-\!x^2\sin^2\varphi}\;d\varphi$ 
$\;=\; \frac{\pi}{2}F(-\frac{1}{2},\,\frac{1}{2},\,1;\,x^2)$\, (complete elliptic integral of 2nd kind)
\end{itemize}
%%%%%
%%%%%
\end{document}
