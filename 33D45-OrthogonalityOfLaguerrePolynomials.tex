\documentclass[12pt]{article}
\usepackage{pmmeta}
\pmcanonicalname{OrthogonalityOfLaguerrePolynomials}
\pmcreated{2013-03-22 19:05:50}
\pmmodified{2013-03-22 19:05:50}
\pmowner{pahio}{2872}
\pmmodifier{pahio}{2872}
\pmtitle{orthogonality of Laguerre polynomials}
\pmrecord{8}{41989}
\pmprivacy{1}
\pmauthor{pahio}{2872}
\pmtype{Derivation}
\pmcomment{trigger rebuild}
\pmclassification{msc}{33D45}
\pmclassification{msc}{33C45}
\pmclassification{msc}{26C05}
%\pmkeywords{orthogonal polynomials}
\pmrelated{SubstitutionNotation}
\pmrelated{PropertiesOfOrthogonalPolynomials}

% this is the default PlanetMath preamble.  as your knowledge
% of TeX increases, you will probably want to edit this, but
% it should be fine as is for beginners.

% almost certainly you want these
\usepackage{amssymb}
\usepackage{amsmath}
\usepackage{amsfonts}

% used for TeXing text within eps files
%\usepackage{psfrag}
% need this for including graphics (\includegraphics)
%\usepackage{graphicx}
% for neatly defining theorems and propositions
%\usepackage{amsthm}
% making logically defined graphics
%%%\usepackage{xypic}

% there are many more packages, add them here as you need them

% define commands here
\newcommand{\sijoitus}[2]%
{\operatornamewithlimits{\Big/}_{\!\!\!#1}^{\,#2}}

\begin{document}
We use the definition of Laguerre polynomials $L_n(x)$ via their \PMlinkname{Rodrigues formula}{RodriguesFormula}
\begin{align}
L_n(x) \;:=\; e^x\frac{d^n}{dx^n}(x^ne^{-x}).
\end{align}
The polynomials (1) themselves are not orthogonal to each other, but the expressions $e^{-\frac{x}{2}}L_n(x)$\, ($n = 0,\,1,\,2,\ldots$) are \PMlinkname{orthogonal}{OrthogonalPolynomials} on the interval from 0 to $\infty$, i.e. the polynomials are orthogonal with respect to the weighting function $e^{-x}$ on that interval, as is seen in the following.\\


Let $m$ be another nonnegative integer.\, We \PMlinkname{integrate by parts}{IntegrationByParts} $m$ times in
$$\int_0^\infty\!e^{-x}x^mL_n(x)\,dx \;=\; \int_0^\infty\!x^m\frac{d^n}{dx^n}(x^ne^{-x})\,dx 
\;=\; (-1)^mm!\int_0^\infty\!\frac{d^{n-m}}{dx^{n-m}}(x^me^{-x})\,dx.$$
When\, $m < n$,\, this yields
\begin{align}
\int_0^\infty\!e^{-x}x^mL_n(x)\,dx \;=\; 
(-1)^mm!\sijoitus{x=0}{\quad \infty}\!\frac{d^{n-m-1}}{dx^{n-m-1}}(x^me^{-x}) \;=\; 0.
\end{align}
and for\, $m = n$\, it gives
\begin{align}
\int_0^\infty\!e^{-x}x^mL_n(x)\,dx \;=\; (-1)^nn!\int_0^\infty\!x^ne^{-x}\,dx \;=\; (-1)^n(n!)^2.
\end{align}
The result (2) implies, because $L_m(x)$ is a polynomial of degree $m$, that
$$\int_0^\infty\!e^{-x}L_m(x)L_n(x)\,dx \;=\; 0 \qquad (m \;<\; n),$$
whence also
\begin{align}
\int_0^\infty\!e^{-x}L_m(x)L_n(x)\,dx \;=\; 0 \qquad (m \;\neq\; n).
\end{align}
Thus the orthogonality has been shown.\, Therefore, since the leading term of $L_n(x)$ is $(-1)^nx^n$, we infer by (3) and (4) that 
$$\int_0^\infty\!e^{-x}[L_n(x)]^2\,dx \;=\; (-1)^n\!\int_0^\infty\!e^{-x}x^nL_n(x)\,ds \;=\; (n!)^2,$$
so that the expressions $\frac{L_n(x)}{n!}$ form a system of orthonormal polynomials.\\

\begin{thebibliography}{9}
\bibitem{QC}{\sc H. Eyring, J. Walter, G. Kimball:} {\em Quantum chemistry}.\, Eight printing.\, Wiley \& Sons, New York (1958).
\end{thebibliography}



%%%%%
%%%%%
\end{document}
