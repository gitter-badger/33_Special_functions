\documentclass[12pt]{article}
\usepackage{pmmeta}
\pmcanonicalname{11nnIsAnIncreasingSequence}
\pmcreated{2013-03-22 15:48:39}
\pmmodified{2013-03-22 15:48:39}
\pmowner{rspuzio}{6075}
\pmmodifier{rspuzio}{6075}
\pmtitle{$(1 + 1/n)^n$ is an increasing sequence}
\pmrecord{14}{37775}
\pmprivacy{1}
\pmauthor{rspuzio}{6075}
\pmtype{Theorem}
\pmcomment{trigger rebuild}
\pmclassification{msc}{33B99}

\endmetadata

% this is the default PlanetMath preamble.  as your knowledge
% of TeX increases, you will probably want to edit this, but
% it should be fine as is for beginners.

% almost certainly you want these
\usepackage{amssymb}
\usepackage{amsmath}
\usepackage{amsfonts}

% used for TeXing text within eps files
%\usepackage{psfrag}
% need this for including graphics (\includegraphics)
%\usepackage{graphicx}
% for neatly defining theorems and propositions
\usepackage{amsthm}
% making logically defined graphics
%%%\usepackage{xypic}

% there are many more packages, add them here as you need them

% define commands here
\newtheorem{thm}{Theorem}

\begin{document}
\begin{thm}
The sequence $(1 + 1/n)^n$ is increasing.
\end{thm}

\begin{proof}
To see this, rewrite $1 + (1/n) = (1 + n)/n$ and divide two
consecutive terms of the sequence:
\begin{eqnarray*}
{ \left( 1 + {1 \over n} \right)^n \over  \left( 1 + {1
\over n-1} \right)^{n-1} } &=&
{ \left( {n+1 \over n} \right)^n \over \left( {n \over n-1}
\right)^{n-1} } \\
&=& \left( {(n-1)(n+1) \over n^2} \right)^{n-1} {n+1 \over n} \\
&=& \left( 1 - {1 \over n^2} \right)^{n-1} \left( 1 + {1 \over n}
\right) \\ 
\end{eqnarray*}

Since $(1 - x)^n \ge 1 - nx$, we have 
\begin{eqnarray*}
{ \left( 1 + {1 \over n} \right)^n \over  \left( 1 + {1
\over n-1} \right)^{n-1} } &\ge&
\left( 1 - {n-1 \over n^2} \right) \left( 1 + {1 \over n}
\right) \\
&=& 1 + {1 \over n^3}\\
&>& 1,
\end{eqnarray*}
hence the sequence is increasing.
\end{proof}

\begin{thm}
The sequence $(1 + 1/n)^{n+1}$ is decreasing.
\end{thm}

\begin{proof}

As before, rewrite $1 + (1/n) = (1 + n)/n$ and divide two
consecutive terms of the sequence:
\begin{eqnarray*}
{ \left( 1 + {1 \over n} \right)^{n+1} \over  \left( 1 + {1
\over n-1} \right)^{n} } &=&
{ \left( {n+1 \over n} \right)^{n+1} \over \left( {n \over n-1}
\right)^{n} } \\
&=& \left( {(n-1)(n+1) \over n^2} \right)^{n} {n+1 \over n} \\
&=& \left( 1 - {1 \over n^2} \right)^{n} \left( 1 + {1 \over n}
\right) \\ 
\end{eqnarray*}

Writing $1 + 1/n$ as $1 + n/n^2$ and applying the inequality
$1 + n/n^2 \le (1 + 1/n^2)^n$, we obtain
\begin{eqnarray*}
{ \left( 1 + {1 \over n} \right)^{n+1} \over  \left( 1 + {1
\over n-1} \right)^{n} } &\le&
\left( 1 - {1 \over n^2} \right)^{n} \left( 1 + {1 \over n^2}
\right)^n \\ 
&=& \left( 1 - {1 \over n^4} \right)^{n} \\
&<& 1, \\
\end{eqnarray*}
hence the sequence is decreasing.

\end{proof}

\begin{thm}
For all positive integers $m$ and $n$, we have $(1 + 1/m)^{m} < (1 + 1/n)^{n+1}$.
\end{thm}

\begin{proof}
We consider three cases.

Suppose that $m = n$.  Since $n > 0$, we have $1/n >0$ and $1 < 1 + 1/n$.  Hence,
$(1 + 1/n)^{n} < (1 + 1/n)^{n+1}$.

Suppose that $m < n$.  By the previous case, $(1 + 1/n)^{n} < (1 + 1/n)^{n+1}$.
By theorem 1, $(1 + 1/m)^{m} < (1 + 1/n)^{n}$.  Combining,
$(1 + 1/m)^{m} < (1 + 1/n)^{n+1}$.

Suppose that $m > n$.
By the first case, $(1 + 1/m)^{m} <(1 + 1/m)^{m+1}$
By theorem 2, $(1 + 1/m)^{m+1} < (1 + 1/n)^{n+1}$.
Combining, $(1 + 1/m)^{m} < (1 + 1/n)^{n+1}$.
\end{proof}

%%%%%
%%%%%
\end{document}
