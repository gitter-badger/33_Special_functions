\documentclass[12pt]{article}
\usepackage{pmmeta}
\pmcanonicalname{LimitsOfNaturalLogarithm}
\pmcreated{2014-12-12 10:15:50}
\pmmodified{2014-12-12 10:15:50}
\pmowner{pahio}{2872}
\pmmodifier{pahio}{2872}
\pmtitle{limits of natural logarithm}
\pmrecord{11}{41859}
\pmprivacy{1}
\pmauthor{pahio}{2872}
\pmtype{Theorem}
\pmcomment{trigger rebuild}
\pmclassification{msc}{33B10}
\pmrelated{ImproperLimits}
\pmrelated{GrowthOfExponentialFunction}
\pmrelated{FundamentalTheoremOfCalculusClassicalVersion}
\pmrelated{DifferentiableFunctionsAreContinuous}

\endmetadata

% this is the default PlanetMath preamble.  as your knowledge
% of TeX increases, you will probably want to edit this, but
% it should be fine as is for beginners.

% almost certainly you want these
\usepackage{amssymb}
\usepackage{amsmath}
\usepackage{amsfonts}

% used for TeXing text within eps files
%\usepackage{psfrag}
% need this for including graphics (\includegraphics)
%\usepackage{graphicx}
% for neatly defining theorems and propositions
 \usepackage{amsthm}
% making logically defined graphics
%%%\usepackage{xypic}

% there are many more packages, add them here as you need them

% define commands here

\theoremstyle{definition}
\newtheorem*{thmplain}{Theorem}

\begin{document}
The \PMlinkname{parent entry}{NaturalLogarithm} defines the natural logarithm as
\begin{align}
\ln{x} \;=\; \int_1^x\frac{1}{t}\,dt \qquad (x > 0)
\end{align}
and derives the \PMlinkescapetext{formula}
$$\ln{xy} \;=\; \ln{x}+\ln{y}$$
which implies easily by induction that
\begin{align}
\ln{a}^n \;=\; n\ln{a}.
\end{align}\\

Basing on (1), we prove here the

\textbf{Theorem.}\, The function \,$x \mapsto \ln{x}$ is strictly increasing and continuous on $\mathbb{R}_+$.\, It has the limits
\begin{align}
\lim_{x\to+\infty}\ln{x} \;=\; +\infty \quad\mbox{and}\quad \lim_{x\to 0+}\ln{x} \;=\; -\infty.
\end{align}


\emph{Proof.}\, By the above definition, $\ln{x}$ is differentiable:
$$\frac{d}{dx}\ln{x} \;=\; \frac{1}{x} \;>\; 0$$
Accordingly, $\ln{x}$ is also continuous and strictly increasing.

Let $M$ be an arbitrary positive number.\, We have\, $\ln2 = \int_1^2\frac{dt}{t} > 0$.\, There exists a positive integer $n$ such that\, $n\ln2 > M$ (see Archimedean property).\, By (2) we thus get\, $\ln{2^n} > M$, and since 
$\ln{x}$ is strictly increasing, we see that
$$\ln{x} > M \quad \forall x > 2^n.$$
Hence the first limit assertion is true.
Now\, $-M < 0$.\, If\, $x > 2^n$,\, then\, $\ln{x} > M$\, and
$$0 \;<\; \frac{1}{x} \;<\; 2^{-n}, \qquad 
\ln\frac{1}{x} \;=\; \int_1^{\frac{1}{x}}\frac{dt}{t} \;=\; \int_x^1\frac{du}{u} \;=\; -\ln{x} \;<\; -M$$
(\PMlinkname{substitution}{SubstitutionForIntegration}\, $xt := u$).\, From this we can infer the second limit assertion.







%%%%%
%%%%%
\end{document}
