\documentclass[12pt]{article}
\usepackage{pmmeta}
\pmcanonicalname{LambertWFunction}
\pmcreated{2013-03-22 12:40:48}
\pmmodified{2013-03-22 12:40:48}
\pmowner{drini}{3}
\pmmodifier{drini}{3}
\pmtitle{Lambert W function}
\pmrecord{8}{32957}
\pmprivacy{1}
\pmauthor{drini}{3}
\pmtype{Definition}
\pmcomment{trigger rebuild}
\pmclassification{msc}{33B30}
\pmsynonym{product log}{LambertWFunction}

\endmetadata

\usepackage{graphicx}
%%%\usepackage{xypic} 
\usepackage{bbm}
\newcommand{\Z}{\mathbbmss{Z}}
\newcommand{\C}{\mathbbmss{C}}
\newcommand{\R}{\mathbbmss{R}}
\newcommand{\Q}{\mathbbmss{Q}}
\newcommand{\mathbb}[1]{\mathbbmss{#1}}
\newcommand{\figura}[1]{\begin{center}\includegraphics{#1}\end{center}}
\newcommand{\figuraex}[2]{\begin{center}\includegraphics[#2]{#1}\end{center}}
\begin{document}
Lambert's $W$ function is the inverse of the function $f: \C \to \C$ given by $f(x) := x e^x$. That is, $W(x)$ is the complex valued function that satisfies
\begin{displaymath}W(x) e^{W(x)} = x,\end{displaymath} 
for all $x \in \mathbb{C}$. In practice the definition of $W(x)$ requires a branch cut, which is usually taken along the negative real axis. Lambert's W function is sometimes also called product log function.

This function allow us to solve the functional equation $$g(x)^{g(x)}=x$$
since $$g(x)=e^{W(\ln(x))}.$$

\section{References}
A site with good information on Lambert's W function is Corless' page 
\PMlinkexternal{``On the Lambert W Function''}{http://kong.apmaths.uwo.ca/~rcorless/frames/PAPERS/LambertW/}
%%%%%
%%%%%
\end{document}
