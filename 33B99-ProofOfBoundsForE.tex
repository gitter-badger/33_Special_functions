\documentclass[12pt]{article}
\usepackage{pmmeta}
\pmcanonicalname{ProofOfBoundsForE}
\pmcreated{2013-03-22 15:48:51}
\pmmodified{2013-03-22 15:48:51}
\pmowner{rspuzio}{6075}
\pmmodifier{rspuzio}{6075}
\pmtitle{proof of bounds for e}
\pmrecord{4}{37779}
\pmprivacy{1}
\pmauthor{rspuzio}{6075}
\pmtype{Proof}
\pmcomment{trigger rebuild}
\pmclassification{msc}{33B99}

\endmetadata

% this is the default PlanetMath preamble.  as your knowledge
% of TeX increases, you will probably want to edit this, but
% it should be fine as is for beginners.

% almost certainly you want these
\usepackage{amssymb}
\usepackage{amsmath}
\usepackage{amsfonts}

% used for TeXing text within eps files
%\usepackage{psfrag}
% need this for including graphics (\includegraphics)
%\usepackage{graphicx}
% for neatly defining theorems and propositions
%\usepackage{amsthm}
% making logically defined graphics
%%%\usepackage{xypic}

% there are many more packages, add them here as you need them

% define commands here
\begin{document}
Multiplying and dividing, we have
\[ \left( 1 + {1 \over n} \right)^n = \left( 1 + {1 \over m} \right)^m
\prod_{k=m+1}^n { \left( 1 + {1 \over k} \right)^k \over \left( 1 + {1
\over k-1} \right)^{k-1}} \]
As was shown in the parent entry, the quotients in the product can
be simplified to give
\[ \left( 1 + {1 \over n} \right)^n = \left( 1 + {1 \over m} \right)^m
\prod_{k=m+1}^n \left( 1 - {1 \over k^2} \right)^k \left( 1 + {1 \over
k-1} \right) \]
By the inequality for differences of powers,
\[ \left( 1 - {1 \over k^2} \right)^k < 1 - {k \over k^2 + k - 1} 
= {(k+1) (k-1) \over k^2 + k - 1} \]
Hence, we have the following upper bound:
\[ \left( 1 + {1 \over n} \right)^n < \left( 1 + {1 \over m} \right)^m
\prod_{k=m+1}^n {k^2 + k \over k^2 + k - 1} \]
By cross mutliplying, it is easy to see that
\[ {k^2 + k \over k^2 + k - 1} \le {k^2 \over k^2 - 1} \]
and, hence,
\[ \left( 1 + {1 \over n} \right)^n < \left( 1 + {1 \over m} \right)^m
\prod_{k=m+1}^n {k^2 \over k^2 - 1}. \]
Factoring the rational function in the product, terms cancel and we
have
\[ \prod_{k=m+1}^n {k^2 \over (k+1)(k-1)} = {n (m+1) \over (n+1)  m}
= {n \over n+1} \left( 1 + {1 \over m} \right) \]
Combining,
\[ \left( 1 + {1 \over n} \right)^n < {n \over n+1} \left( 1
+ {1 \over m} \right)^{m+1} \]
%%%%%
%%%%%
\end{document}
