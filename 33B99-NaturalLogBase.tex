\documentclass[12pt]{article}
\usepackage{pmmeta}
\pmcanonicalname{NaturalLogBase}
\pmcreated{2013-03-22 11:55:56}
\pmmodified{2013-03-22 11:55:56}
\pmowner{CWoo}{3771}
\pmmodifier{CWoo}{3771}
\pmtitle{natural log base}
\pmrecord{11}{30657}
\pmprivacy{1}
\pmauthor{CWoo}{3771}
\pmtype{Definition}
\pmcomment{trigger rebuild}
\pmclassification{msc}{33B99}
\pmsynonym{Euler number}{NaturalLogBase}
\pmsynonym{Eulerian number}{NaturalLogBase}
\pmsynonym{Napier's constant}{NaturalLogBase}
\pmsynonym{e}{NaturalLogBase}
\pmrelated{ExampleOfTaylorPolynomialsForTheExponentialFunction}
\pmrelated{EIsTranscendental}
\pmrelated{EIsIrrationalProof}
\pmrelated{ApplicationOfCauchyCriterionForConvergence}

\endmetadata

\usepackage{amssymb}
\usepackage{amsmath}
\usepackage{amsfonts}
\usepackage{graphicx}
%%%%\usepackage{xypic}
\begin{document}
The \emph{natural log base}, or $e$, has value 

\[ 2.718281828459045\ldots \]

$e$ was extensively studied by Euler in the 1720's, but it was originally discovered by John Napier. 

$e$ is defined by 

\[ \lim_{n \rightarrow \infty} \left(1+\frac{1}{n}\right)^n \]

It is more effectively calculated, however, by using the Taylor series for $f(x)=e^x$ at $x=1$ to get the representation 

\[ e = \frac{1}{0!} + \frac{1}{1!} + \frac{1}{2!} + \frac{1}{3!} + \frac{1}{4!} + \cdots \]
%%%%%
%%%%%
%%%%%
%%%%%
\end{document}
