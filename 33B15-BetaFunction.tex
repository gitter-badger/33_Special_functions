\documentclass[12pt]{article}
\usepackage{pmmeta}
\pmcanonicalname{BetaFunction}
\pmcreated{2013-03-22 13:26:23}
\pmmodified{2013-03-22 13:26:23}
\pmowner{yark}{2760}
\pmmodifier{yark}{2760}
\pmtitle{beta function}
\pmrecord{21}{34001}
\pmprivacy{1}
\pmauthor{yark}{2760}
\pmtype{Definition}
\pmcomment{trigger rebuild}
\pmclassification{msc}{33B15}

\endmetadata

\usepackage{amssymb}
\usepackage{amsmath}
\usepackage{amsfonts}

\begin{document}
\PMlinkescapeword{property}
\PMlinkescapeword{side}

The \emph{beta function} is defined as
\[
  B(p,q) = \int_0^1 x^{p-1} (1-x)^{q-1} dx
\]
for any real numbers $p,q > 0$.
For other complex values of $p$ and $q$,
we can define $B(p,q)$ by analytic continuation.

The beta function has the property
\[
  B(p,q) = \frac{\Gamma(p) \Gamma(q)}{\Gamma(p+q)}
\]
for all complex numbers $p$ and $q$ for which the right-hand side is defined.
Here $\Gamma$ is the gamma function.

Also,
\[
  B(p,q) = B(q,p)
\]
and
\[
  B({\textstyle\frac{1}{2},\frac{1}{2}}) = \pi.
\]

The beta function was first defined
by \PMlinkname{L.~Euler}{EulerLeonhard} in 1730,
and the name was given by J.~Binet.
%%%%%
%%%%%
\end{document}
