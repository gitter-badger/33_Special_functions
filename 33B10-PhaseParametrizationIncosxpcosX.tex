\documentclass[12pt]{article}
\usepackage{pmmeta}
\pmcanonicalname{PhaseParametrizationIncosxpcosX}
\pmcreated{2013-03-22 17:56:15}
\pmmodified{2013-03-22 17:56:15}
\pmowner{perucho}{2192}
\pmmodifier{perucho}{2192}
\pmtitle{phase parametrization in $\cos(x+p)=\cos x$}
\pmrecord{4}{40434}
\pmprivacy{1}
\pmauthor{perucho}{2192}
\pmtype{Definition}
\pmcomment{trigger rebuild}
\pmclassification{msc}{33B10}
\pmclassification{msc}{42-00}
\pmclassification{msc}{51-00}
\pmclassification{msc}{43-00}

% this is the default PlanetMath preamble.  as your knowledge
% of TeX increases, you will probably want to edit this, but
% it should be fine as is for beginners.

% almost certainly you want these
\usepackage{amssymb}
\usepackage{amsmath}
\usepackage{amsfonts}
\usepackage{amsthm}

% used for TeXing text within eps files
%\usepackage{psfrag}
% need this for including graphics (\includegraphics)
%\usepackage{graphicx}
% for neatly defining theorems and propositions
%\usepackage{amsthm}
% making logically defined graphics
%%%\usepackage{xypic}
\usepackage{pstricks}
\usepackage{pst-plot}

% there are many more packages, add them here as you need them

% define commands here
\newtheorem{theorem}{Theorem}
\newtheorem{defn}{Definition}
\newtheorem{prop}{Proposition}
\newtheorem{lemma}{Lemma}
\newtheorem{cor}{Corollary}

\begin{document}
We obtain a single equation to parametrization of phase's angle $p$ for the periodic cosine function when the typical condition $\cos(x+p)=cos x$ is satisfied. Without losing generality, we will restrict the discussion to the case $0\leq x \leq 2\pi$. We starting setting $\cos x=t$, so that $-1\leq t\leq 1$ is the interval of definition for the real parameter $t$. Next we expand out
\begin{equation*}
\cos(x+p)=\cos x\cos p-\sin x\sin p,
\end{equation*}
where we introduce $t$ and apply the assumed condition to get
\begin{equation*}
t\cos p-\sqrt{1-t^2}\sin p=t.
\end{equation*} 
This leads to a quadratic equation in $\cos p$ on terms of parameter $t$, i.e.
\begin{equation*}
\cos^2p(t)-2t^2\cos p(t)+(2t^2-1)=0.
\end{equation*}
Solving,
\begin{equation*}
\cos p(t)=t^2\pm (t^2-1).
\end{equation*}
One root has to do with the trivial $\cos(x+2\pi)=\cos x$, but we are interested on the nontrivial one
\begin{equation}
\cos p(t)=2t^2-1, \qquad -1<t<1.
\end{equation}
Locus of (1) is the below shown parabola. \\ \\


\begin{center}
\psset{unit=2cm}
\begin{pspicture}(-2.5,-2)(2.5,5)
\psaxes[Dx=1,Dy=1]{->}(0,0)(-2.2,-1.5)(2.3,2.5)
\rput(2.3,-0.2){$t$}
\rput(0.2,2.6){$y$}
\psplot[linecolor=green]{-1}{-0.707}{2 x x mul mul 1 sub}
\psplot[linecolor=red]{-0.707}{0.707}{2 x x mul mul 1 sub}
\psplot[linecolor=green]{0.707}{1}{2 x x mul mul 1 sub}
\rput(1.1,1.3){$y = 2t^2-1$}
\psdot(-0.707,0)
\psdot(0.707,0)
\psdot(-1,1)
\psdot(1,1)
\end{pspicture}
\end{center}


Inverse function of (1) is the wanted parametrization $p(t)$, with codomain $[0,\pi]$ and locus the \emph{``arabian dome''} below shown.{\footnote{Graphics by \emph{pahio}, with permission.}} \\ \\

\begin{center}
\psset{unit=2cm}
\begin{pspicture}(-2.5,-2)(2.5,4)
\psaxes[Dx=1,Dy=1]{->}(0,0)(-2.2,-1.5)(2.3,3.7)
\rput(2.3,-0.2){$t$}
\rput(0.2,3.75){$p$}
\psplot[linecolor=blue]{-1}{1}{2 x mul x mul 1 sub arccos 3.1416 mul 180 div}
\rput(1.8,2.2){$p = \arccos(2t^2-1)$}
\end{pspicture}
\end{center}





%%%%%
%%%%%
\end{document}
