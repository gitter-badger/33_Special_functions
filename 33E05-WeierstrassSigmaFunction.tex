\documentclass[12pt]{article}
\usepackage{pmmeta}
\pmcanonicalname{WeierstrassSigmaFunction}
\pmcreated{2013-03-22 13:54:06}
\pmmodified{2013-03-22 13:54:06}
\pmowner{alozano}{2414}
\pmmodifier{alozano}{2414}
\pmtitle{Weierstrass sigma function}
\pmrecord{4}{34650}
\pmprivacy{1}
\pmauthor{alozano}{2414}
\pmtype{Definition}
\pmcomment{trigger rebuild}
\pmclassification{msc}{33E05}
\pmsynonym{sigma function}{WeierstrassSigmaFunction}
\pmsynonym{zeta function}{WeierstrassSigmaFunction}
\pmsynonym{eta function}{WeierstrassSigmaFunction}
%\pmkeywords{Weierstrass}
%\pmkeywords{sigma}
%\pmkeywords{eta}
%\pmkeywords{zeta}
\pmrelated{EllipticFunction}
\pmrelated{ModularDiscriminant}
\pmdefines{Weierstrass sigma function}
\pmdefines{Weierstrass zeta function}
\pmdefines{Weierstrass eta function}

% this is the default PlanetMath preamble.  as your knowledge
% of TeX increases, you will probably want to edit this, but
% it should be fine as is for beginners.

% almost certainly you want these
\usepackage{amssymb}
\usepackage{amsmath}
\usepackage{amsthm}
\usepackage{amsfonts}

% used for TeXing text within eps files
%\usepackage{psfrag}
% need this for including graphics (\includegraphics)
%\usepackage{graphicx}
% for neatly defining theorems and propositions
%\usepackage{amsthm}
% making logically defined graphics
%%%\usepackage{xypic}

% there are many more packages, add them here as you need them

% define commands here

\newtheorem{thm}{Theorem}
\newtheorem{defn}{Definition}
\newtheorem{prop}{Proposition}
\newtheorem{lemma}{Lemma}
\newtheorem{cor}{Corollary}

% Some sets
\newcommand{\Nats}{\mathbb{N}}
\newcommand{\Ints}{\mathbb{Z}}
\newcommand{\Reals}{\mathbb{R}}
\newcommand{\Complex}{\mathbb{C}}
\newcommand{\Rats}{\mathbb{Q}}
\begin{document}
\begin{defn}
Let $\Lambda\subset\Complex$ be a lattice. Let $\Lambda^{\ast}$
denote $\Lambda-\{ 0 \}$.
\begin{enumerate}
\item The \emph{Weierstrass sigma function} is defined as the
product
$$\sigma(z;\Lambda)=z\prod_{w\in\Lambda^{\ast}}\left(1-\frac{z}{w}\right)e^{z/w+\frac{1}{2}(z/w)^2}$$

\item The \emph{Weierstrass zeta function} is defined by the sum
$$\zeta(z;\Lambda)=\frac{\sigma'(z;\Lambda)}{\sigma(z;\Lambda)}=\frac{1}{z}+\sum_{w\in\Lambda^{\ast}}\left( \frac{1}{z-w}+\frac{1}{w}+\frac{z}{w^2}\right)$$
Note that the Weierstrass zeta function is basically the
derivative of the logarithm of the sigma function. The zeta
function can be rewritten as:
$$\zeta(z;\Lambda)=\frac{1}{z}-\sum_{k=1}^{\infty}\mathcal{G}_{2k+2}(\Lambda)z^{2k+1}$$
where $\mathcal{G}_{2k+2}$ is the Eisenstein series of weight
$2k+2$.

\item The \emph{Weierstrass eta function} is defined to be
$$\eta(w;\Lambda)=\zeta(z+w;\Lambda)-\zeta(z;\Lambda), \text{for
any } z\in\Complex$$ (It can be proved that this is well defined,
i.e. $\zeta(z+w;\Lambda)-\zeta(z;\Lambda)$ only depends on $w$).
The Weierstrass eta function must not be confused with the
Dedekind eta function.
\end{enumerate}
\end{defn}
%%%%%
%%%%%
\end{document}
