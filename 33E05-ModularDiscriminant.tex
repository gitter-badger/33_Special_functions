\documentclass[12pt]{article}
\usepackage{pmmeta}
\pmcanonicalname{ModularDiscriminant}
\pmcreated{2013-03-22 13:54:09}
\pmmodified{2013-03-22 13:54:09}
\pmowner{alozano}{2414}
\pmmodifier{alozano}{2414}
\pmtitle{modular discriminant}
\pmrecord{6}{34651}
\pmprivacy{1}
\pmauthor{alozano}{2414}
\pmtype{Definition}
\pmcomment{trigger rebuild}
\pmclassification{msc}{33E05}
\pmsynonym{delta function}{ModularDiscriminant}
\pmrelated{EllipticFunction}
\pmrelated{JInvariant}
\pmrelated{WeierstrassSigmaFunction}
\pmrelated{Discriminant}
\pmrelated{DiscriminantOfANumberField}
\pmrelated{RamanujanTauFunction}
\pmdefines{modular discriminant}
\pmdefines{Dedekind eta function}

% this is the default PlanetMath preamble.  as your knowledge
% of TeX increases, you will probably want to edit this, but
% it should be fine as is for beginners.

% almost certainly you want these
\usepackage{amssymb}
\usepackage{amsmath}
\usepackage{amsthm}
\usepackage{amsfonts}

% used for TeXing text within eps files
%\usepackage{psfrag}
% need this for including graphics (\includegraphics)
%\usepackage{graphicx}
% for neatly defining theorems and propositions
%\usepackage{amsthm}
% making logically defined graphics
%%%\usepackage{xypic}

% there are many more packages, add them here as you need them

% define commands here

\newtheorem{thm}{Theorem}
\newtheorem{defn}{Definition}
\newtheorem{prop}{Proposition}
\newtheorem{lemma}{Lemma}
\newtheorem{cor}{Corollary}

% Some sets
\newcommand{\Nats}{\mathbb{N}}
\newcommand{\Ints}{\mathbb{Z}}
\newcommand{\Reals}{\mathbb{R}}
\newcommand{\Complex}{\mathbb{C}}
\newcommand{\Rats}{\mathbb{Q}}
\begin{document}
\begin{defn}
Let $\Lambda\subset\Complex$ be a lattice.
\begin{enumerate}
\item Let $q_{\tau}=e^{2\pi i \tau}$. The \emph{Dedekind eta
function} is defined to be
$$\eta(\tau)=q_{\tau}^{1/24}\prod_{n=1}^{\infty}(1-q_{\tau}^n)$$
The Dedekind eta function should not be confused with the
Weierstrass eta function, $\eta(w;\Lambda)$.

\item The $j$-invariant, as a function of lattices, is defined to
be:
$$j(\Lambda)=\frac{g_2^3}{g_2^3-27g_3^2}$$
where $g_2$ and $g_3$ are certain multiples of the Eisenstein
series of weight $4$ and $6$ (see \PMlinkexternal{this
entry}{http://planetmath.org/encyclopedia/ExamplesOfEllipticFunctions.html}).

\item The \emph{$\Delta$ function} (\emph{delta function} or
\emph{modular discriminant}) is defined to be
$$\Delta(\Lambda)=g_2^3-27g_3^2$$
Let $\Lambda_{\tau}$ be the lattice generated by $1,\tau$. The $\Delta$ function for $\Lambda_{\tau}$ has a product expansion
$$\Delta(\tau)=\Delta(\Lambda_{\tau})=(2\pi
i)^{12}q_{\tau}\prod_{n=1}^{\infty}(1-q_{\tau}^n)^{24}=(2\pi
i)^{12}\eta(\tau)^{24}$$

\end{enumerate}
\end{defn}
%%%%%
%%%%%
\end{document}
