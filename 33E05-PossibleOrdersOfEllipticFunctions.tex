\documentclass[12pt]{article}
\usepackage{pmmeta}
\pmcanonicalname{PossibleOrdersOfEllipticFunctions}
\pmcreated{2013-03-22 15:48:30}
\pmmodified{2013-03-22 15:48:30}
\pmowner{rspuzio}{6075}
\pmmodifier{rspuzio}{6075}
\pmtitle{possible orders of elliptic functions}
\pmrecord{7}{37772}
\pmprivacy{1}
\pmauthor{rspuzio}{6075}
\pmtype{Theorem}
\pmcomment{trigger rebuild}
\pmclassification{msc}{33E05}
\pmrelated{PeriodicFunctions}

% this is the default PlanetMath preamble.  as your knowledge
% of TeX increases, you will probably want to edit this, but
% it should be fine as is for beginners.

% almost certainly you want these
\usepackage{amssymb}
\usepackage{amsmath}
\usepackage{amsfonts}

% used for TeXing text within eps files
%\usepackage{psfrag}
% need this for including graphics (\includegraphics)
%\usepackage{graphicx}
% for neatly defining theorems and propositions
%\usepackage{amsthm}
% making logically defined graphics
%%%\usepackage{xypic}

% there are many more packages, add them here as you need them

% define commands here
\begin{document}
The order of a non-trivial elliptic function cannot be zero.  This is a 
simple consequence of Liouville's theorem.  Were the order of an elliptic
function zero, then the function would have no poles.  By definition,
an elliptic function has no essential singularities and is doubly
periodic.  Hence, if the degree were zero, the function would be
continuous everywhere and hence, being doubly periodic, would be
bounded (since continuous functions on a compact domain (like the
closure of the fundamental parallelogram) are bounded).  By Liouville's
theorem, this would imply that the function is constant.

The order of an elliptic function cannot be 1.  This follows from the
fact that the residues at the poles of an elliptic function within a
fundamental parallelogram must sum to zero --- if the function were of
degree 1, it would have exactly one first-order pole in the
fundamental parallelgram but any first-order pole must have a non-zero
residue.

Any number greater than one is possible as the order of an elliptic
function.  As an example of an elliptic function of order two, we may
take the Weierstass $\wp$-function, which has a single pole of order 2
in the fundamental domain.  The $n$-th derivative of this function
will have a single pole of order $n+2$ in the fundamental domain,
hence be of order $n+2$, so is an example showing that, for every
integer greater than 2, there exists an elliptic function having that
integer as its order.
%%%%%
%%%%%
\end{document}
