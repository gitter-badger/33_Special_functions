\documentclass[12pt]{article}
\usepackage{pmmeta}
\pmcanonicalname{ReductionOfEllipticIntegralsToStandardForm}
\pmcreated{2014-02-01 18:13:38}
\pmmodified{2014-02-01 18:13:38}
\pmowner{rspuzio}{6075}
\pmmodifier{rspuzio}{6075}
\pmtitle{reduction of elliptic integrals to standard form}
\pmrecord{30}{38610}
\pmprivacy{1}
\pmauthor{rspuzio}{6075}
\pmtype{Theorem}
\pmcomment{trigger rebuild}
\pmclassification{msc}{33E05}
\pmrelated{ExpressibleInClosedForm}

% this is the default PlanetMath preamble.  as your knowledge
% of TeX increases, you will probably want to edit this, but
% it should be fine as is for beginners.

% almost certainly you want these
\usepackage{amssymb}
\usepackage{amsmath}
\usepackage{amsfonts}

% used for TeXing text within eps files
%\usepackage{psfrag}
% need this for including graphics (\includegraphics)
%\usepackage{graphicx}
% for neatly defining theorems and propositions
%\usepackage{amsthm}
% making logically defined graphics
%%%\usepackage{xypic}

% there are many more packages, add them here as you need them

% define commands here

\begin{document}
Any integral of the form $\int R(x,\sqrt{P(x)}) \, dx$, where $R$ is a rational 
function and $P$ is a polynomial of degree 3 or 4 can be expressed as a linear 
combination of elementary functions and elliptic integrals of the first, second, 
and third kinds.

To begin, we will assume that $P$ has no repeated roots.  Were this not the case, 
we could simply pull the repeated factor out of the radical and be left with a 
polynomial of degree of 1 or 2 inside the square root and express the integral in 
terms of inverse trigonometric functions.

Make a change of variables $z = (a x + b) / (c x + d)$.  By choosing the coefficients 
$a, b, c, d$ suitably, one can cast P into either Jacobi's normal form 
$P(z) = (1 - z^2) (1 - k^2 z^2)$ or Weierstrass' normal form 
$P(z) = 4 z^3 - g_2 z - g_3$.

Note that 
 \[R (z, \sqrt{P(z)}) = \frac{A(z) + B(z) \sqrt{P(z)}}{C(z) + D(z) \sqrt{P(z)}}\]
for suitable polynomials $A, B, C, D$.  We can rationalize the denominator like so:
 \[\frac{A(z) + B(z) \sqrt{P(z)}}{C(z) + D(z) \sqrt{P(z)}} \times \frac{C(z) -
D(z) \sqrt{P(z)}}{C(z) - D(z) \sqrt{P(z)}} = F(z) + G(z) \sqrt{P(z)}\]
The rational functions $F$ and $G$ appearing in the foregoing equation are defined like so:
\begin{eqnarray*} F(z) &=& \frac{A(z) C(z) - B(z) D(z) P(z)}{C^2 (z) - D^2(z) P(z)} \\ G(z) &=& 2 \frac{B(z) C(z) - A(z) D(z)}{C^2 (z) - D^2(z) P(z)} \end{eqnarray*}

Since $\int F(z) \, dz$ may be expressed in terms of elementary functions, 
we shall focus our attention on the remaining piece, $\int G(z) \sqrt{P(z)} \, dz$,
which we shall write as $\int H(z) / \sqrt{P(z)} \, dz$, where $H = PG$.. 
Because we may decompose $H$ into partial fractions, it suffices to consider
the following cases, which we shall all $A_n$ and $B_n$: 
 \[ A_n(z) = \int \frac{z^n}{\sqrt{P(z)}} \, dz\]
 \[ B_n(z,r) = \int \frac{1}{(z - r)^n \sqrt{P(z)}} \, dz\]
Here, $n$ is a non-negative integer and $r$ is a complex number.

We will reduce thes further using integration by parts.  
Taking antiderivatives, we have:
 \[ \int \frac{z^{n-1} (zP'(z) + 2nP(z))}{2\sqrt{P(z)}} \, dz = z^n \sqrt{P(z)} + C \]
 \[  \int \frac{(z-r) P'(z) - 2nP(z)}{2(z-r)^{n+1} \sqrt{P(z)}} \, dz
 = \frac{\sqrt{P(z)}}{(z-r)^n} + C \]
 These identities will allow us to express $A_n$'s and $B_n$'s with large
 $n$ in terms of ones with smaller $n$'s.

At this point, it is convenient to employ the specific form of the polynominal $P$.  
We will first conside the Weierstrass normal form and then the Jacobi normal form.

Substituting into our identities and collecting terms, we find 
 \[ 4 (2n + 3) A_{n+2} = (2n + 1) g_2 A_n + 2n g_3 A_{n-1} + z^n \sqrt{4z^3 - g_2 x - g_3} + C \] 

\[ 2n (4r^3 - g_2 r - g_3) B_{n+1} + (2n - 1) (12 r^2 - g_2) B_n 
  + 24 (n - 1) r B_{n-1} + 4(2n - 3) B_{n-2} + \frac{\sqrt{4z^3 - g_2 x - g_3}}{(z-r)^n} + C = 0 \]

Note that there are some cases which can be integrated in elementary terms.  Namely, suppose that the power is odd:
 \[\int z^{2m+1} \sqrt{(1 - z^2) (1 - k^2 z^2)} \, dz\]
Then we may make a change of variables $y = z^2$ to obtain
 \[\frac{1}{2} \int y^{2m} \sqrt{(1 - y) (1 - k^2 y)} \, dy ,\]
which may be integrated using elementary functions. 

Next, we derive some identities using integration by parts.  Since
\[ d \left( (1 - z^2) (1 - k^2 z^2) \sqrt{(1 - z^2) (1 - k^2 z^2)} \right) 
= 
\left( \frac{9}{2} k^2 z^3 - 3 (1 + k^2) z \right)
\sqrt{(1 - z^2) (1 - k^2 z^2)} \, dz , \]
we have
\begin{eqnarray*} (2 m + 1) &&\int z^{2m} (1 - z^2) (1 - k^2 z^2) \sqrt{(1 - z^2) (1 - 
k^2 z^2)} \, dz \\ + &&\int z^{2m+1} \left( \frac{9}{2} k^2 z^3 - 3 (1 + k^2) z \right)
\sqrt{(1 - z^2) (1 - k^2 z^2)} \, dz \\
= && z^{2m+1} (1 - z^2) (1 - k^2 z^2) \sqrt{(1 - z^2) (1 - k^2 z^2)} + C 
\end{eqnarray*}
By colecting terms, this identity may be rewritten as follows:
\begin{eqnarray*} \left( 1 + 2 m + \frac{9}{2} k^2 \right) &&\int z^{2m+4} \sqrt{(1 - z^2) (1 - k^2 z^2)} \, dz  - \\ (4 + 2 m) (1 + k^2) &&\int z^{2m+2} \sqrt{(1 - z^2) (1 - k^2 z^2)} \, dz + \\ &&\int z^{2m} \sqrt{(1 - z^2) (1 - k^2 z^2)} = \\ && x^{2k+1} (1 - z^2) (1 - k^2 z^2) \sqrt{(1 - z^2) (1 - k^2 z^2)} + C 
\end{eqnarray*}
By repeated use of this identity, we may express any integral of the form $\int z^{2m} \sqrt{P(z)} \, dz$ as the sum of a linear combination of $\int z^2 \sqrt{P(z)} \, dz$ and $\int \sqrt{P(z)} \, dz$ and the product of a polyomial and $\sqrt{P(z)}$.

Likewise, we can use integration by parts to simplify integrals of the form 
 \[\int \frac{\sqrt{P(z)}}{(z - r)^n} \, dz\]

\emph{ Will finish later --- saving in case of computer crash.}
%%%%%
%%%%%
\end{document}
