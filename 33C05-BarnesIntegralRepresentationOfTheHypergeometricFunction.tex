\documentclass[12pt]{article}
\usepackage{pmmeta}
\pmcanonicalname{BarnesIntegralRepresentationOfTheHypergeometricFunction}
\pmcreated{2013-03-22 17:36:15}
\pmmodified{2013-03-22 17:36:15}
\pmowner{rspuzio}{6075}
\pmmodifier{rspuzio}{6075}
\pmtitle{Barnes' integral representation of the  hypergeometric function}
\pmrecord{4}{40019}
\pmprivacy{1}
\pmauthor{rspuzio}{6075}
\pmtype{Theorem}
\pmcomment{trigger rebuild}
\pmclassification{msc}{33C05}

\endmetadata

% this is the default PlanetMath preamble.  as your knowledge
% of TeX increases, you will probably want to edit this, but
% it should be fine as is for beginners.

% almost certainly you want these
\usepackage{amssymb}
\usepackage{amsmath}
\usepackage{amsfonts}

% used for TeXing text within eps files
%\usepackage{psfrag}
% need this for including graphics (\includegraphics)
%\usepackage{graphicx}
% for neatly defining theorems and propositions
%\usepackage{amsthm}
% making logically defined graphics
%%%\usepackage{xypic}

% there are many more packages, add them here as you need them

% define commands here

\begin{document}
When $a,b,c,d$ are complex numbers and $z$ is a complex number 
such that $-\pi < \arg (-z) < +\pi$ and $C$ is a contour in the 
complex $s$-plane which goes from $-i \infty$ to $+ i \infty$ 
chosen such that the poles of $\Gamma (a+s) \Gamma (b+s)$ lie
to the left of $C$ and the poles of $\Gamma (-s)$ lie to the
right of $C$, then
\[
\int_C {\Gamma (a+s) \Gamma (b+s) \over \Gamma (c+s)}
\Gamma (-s) (-z)^s \, ds = 
2 \pi i {\Gamma (a) \Gamma (b) \over \Gamma (c)} F (a, b; c; z)
\]
%%%%%
%%%%%
\end{document}
