\documentclass[12pt]{article}
\usepackage{pmmeta}
\pmcanonicalname{EvaluationOfBetaFunctionUsingLaplaceTransform}
\pmcreated{2013-03-22 14:37:36}
\pmmodified{2013-03-22 14:37:36}
\pmowner{rspuzio}{6075}
\pmmodifier{rspuzio}{6075}
\pmtitle{evaluation of beta function using Laplace transform}
\pmrecord{10}{36206}
\pmprivacy{1}
\pmauthor{rspuzio}{6075}
\pmtype{Derivation}
\pmcomment{trigger rebuild}
\pmclassification{msc}{33B15}

% this is the default PlanetMath preamble.  as your knowledge
% of TeX increases, you will probably want to edit this, but
% it should be fine as is for beginners.

% almost certainly you want these
\usepackage{amssymb}
\usepackage{amsmath}
\usepackage{amsfonts}

% used for TeXing text within eps files
%\usepackage{psfrag}
% need this for including graphics (\includegraphics)
%\usepackage{graphicx}
% for neatly defining theorems and propositions
%\usepackage{amsthm}
% making logically defined graphics
%%%\usepackage{xypic}

% there are many more packages, add them here as you need them

% define commands here
\begin{document}
The beta integral can be evaluated elegantly using the \PMlinkname{convolution theorem}{LaplaceTransform} for Laplace transforms.

Start with the following Laplace transform:
\[
s^{-\alpha} = {\cal L} \left[ {t^{\alpha - 1} \over \Gamma(\alpha)} \right] = \int_0^\infty e^{-st} {t^{\alpha - 1} \over \Gamma(\alpha)} dt
\]

Since $s^{-q} s^{-p} = s^{-q - p}$, the convolution theorem imples that
\[
{t^{q - 1} \over \Gamma(q)} *  {t^{p - 1} \over \Gamma(p)} =  {t^{q + p - 1} \over \Gamma(q + p)}
\]

Writing out the definition of convolution, this becomes
\[
\int_0^t {(t-s)^{q - 1} \over \Gamma(q)} {s^{p - 1} \over \Gamma(p)} ds = {t^{q + p - 1} \over \Gamma(p + q)}
\]

Setting $t=1$ and simplifying, we conclude that
\[
\int_0^1 x^{p - 1} (1-x)^{q - 1}  \,dx= {\Gamma(p) \Gamma(q)  \over \Gamma(p + q)}
\]

\rightline{QED}
%%%%%
%%%%%
\end{document}
