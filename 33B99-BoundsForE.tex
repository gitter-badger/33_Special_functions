\documentclass[12pt]{article}
\usepackage{pmmeta}
\pmcanonicalname{BoundsForE}
\pmcreated{2013-03-22 15:48:48}
\pmmodified{2013-03-22 15:48:48}
\pmowner{rspuzio}{6075}
\pmmodifier{rspuzio}{6075}
\pmtitle{bounds for e}
\pmrecord{7}{37778}
\pmprivacy{1}
\pmauthor{rspuzio}{6075}
\pmtype{Theorem}
\pmcomment{trigger rebuild}
\pmclassification{msc}{33B99}

\endmetadata

% this is the default PlanetMath preamble.  as your knowledge
% of TeX increases, you will probably want to edit this, but
% it should be fine as is for beginners.

% almost certainly you want these
\usepackage{amssymb}
\usepackage{amsmath}
\usepackage{amsfonts}

% used for TeXing text within eps files
%\usepackage{psfrag}
% need this for including graphics (\includegraphics)
%\usepackage{graphicx}
% for neatly defining theorems and propositions
%\usepackage{amsthm}
% making logically defined graphics
%%%\usepackage{xypic}

% there are many more packages, add them here as you need them

% define commands here
\begin{document}
If $n$ and $m$ are positive integers and $n > m$, we have
the following inequality:
\[ \left( 1 + {1 \over n} \right)^n < {n \over n+1} \left( 1
+ {1 \over m} \right)^{m+1} \]

Taking the limit as $n \to \infty$, we obtain an upper bound for $e$.
Combining this with the fact that the $(1 + 1/n)^n$ is an increasing
sequence, we have the following bounds for $e$:
\[ \left( 1 + {1 \over m} \right)^m < e < \left( 1
+ {1 \over m} \right)^{m+1}\]

This can be used to show that $e$ is not an integer -- if we take
$m = 5$, we obtain $2.48832 < e < 2.985984$, for instance.
%%%%%
%%%%%
\end{document}
