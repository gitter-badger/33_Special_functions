\documentclass[12pt]{article}
\usepackage{pmmeta}
\pmcanonicalname{DifferentialdifferenceEquationsForHypergeometricFunction}
\pmcreated{2013-03-22 17:36:18}
\pmmodified{2013-03-22 17:36:18}
\pmowner{rspuzio}{6075}
\pmmodifier{rspuzio}{6075}
\pmtitle{differential-difference equations for hypergeometric function}
\pmrecord{6}{40020}
\pmprivacy{1}
\pmauthor{rspuzio}{6075}
\pmtype{Theorem}
\pmcomment{trigger rebuild}
\pmclassification{msc}{33C05}

% this is the default PlanetMath preamble.  as your knowledge
% of TeX increases, you will probably want to edit this, but
% it should be fine as is for beginners.

% almost certainly you want these
\usepackage{amssymb}
\usepackage{amsmath}
\usepackage{amsfonts}

% used for TeXing text within eps files
%\usepackage{psfrag}
% need this for including graphics (\includegraphics)
%\usepackage{graphicx}
% for neatly defining theorems and propositions
%\usepackage{amsthm}
% making logically defined graphics
%%%\usepackage{xypic}

% there are many more packages, add them here as you need them

% define commands here

\begin{document}
The hypergeometric function satisfies several equations which
relate derivatives with respect to the argument $z$ to
shifting the parameters $a, b, c, d$ by unity (Here, the
prime denotes derivative with respect to $z$.):
\begin{align*}
z F' (a, b; c; z) + a F (a, b; c; z) &= F (a+1, b; c; z) \\
z F' (a, b; c; z) + b F (a, b; c; z) &= F (a, b+1; c; z) \\
z F' (a, b; c; z) + (c-1) F (a, b; c; z) &= F (a, b; c-1; z) \\
(1 - z) z F' (a, b; c; z) &= (c - a) F (a-1, b; c; z) + (a - c + bz) F (a, b; c; z) \\
(1 - z) z F' (a, b; c; z) &= (c - b) F (a, b-1; c; z) + (b - c + az) F (a, b; c; z) \\
(1 - z) z F' (a, b; c; z) &= z (c - a) (c - b) F (a, b; c+1; z) +
zc (a + b - c) F (a, b; c; z)
\end{align*}
These equations may readily be verified by differentiating the series
which defines the hypergeometric equation.  By eliminating the derivatives
between these equations, one obtains the contiguity relations for the 
hypergeometric function.  By differentiating them once more and taking
suitable linear combinations, one may obtain the differential equation
of the hypergeometric function.
%%%%%
%%%%%
\end{document}
