\documentclass[12pt]{article}
\usepackage{pmmeta}
\pmcanonicalname{ValuesOfdisplaystylesumi0nfrac1iFor0N26}
\pmcreated{2013-03-22 17:06:13}
\pmmodified{2013-03-22 17:06:13}
\pmowner{PrimeFan}{13766}
\pmmodifier{PrimeFan}{13766}
\pmtitle{values of $\displaystyle \sum_{i = 0}^n \frac{1}{i!}$ for $0 < n < 26$}
\pmrecord{5}{39401}
\pmprivacy{1}
\pmauthor{PrimeFan}{13766}
\pmtype{Example}
\pmcomment{trigger rebuild}
\pmclassification{msc}{33B99}

% this is the default PlanetMath preamble.  as your knowledge
% of TeX increases, you will probably want to edit this, but
% it should be fine as is for beginners.

% almost certainly you want these
\usepackage{amssymb}
\usepackage{amsmath}
\usepackage{amsfonts}

% used for TeXing text within eps files
%\usepackage{psfrag}
% need this for including graphics (\includegraphics)
%\usepackage{graphicx}
% for neatly defining theorems and propositions
%\usepackage{amsthm}
% making logically defined graphics
%%%\usepackage{xypic}

% there are many more packages, add them here as you need them

% define commands here

\begin{document}
The following table gives the numerator and denominator of $$\sum_{i = 0}^n \frac{1}{i!}$$ as well as the decimal expansion to 20 places.

\begin{tabular}{|r|r|r|l|}
$n$ & Numerator of $\displaystyle \sum_{i = 0}^n \frac{1}{i!}$ & Denominator of $\displaystyle \sum_{i = 0}^n \frac{1}{i!}$ & Decimal value of $\displaystyle \sum_{i = 0}^n \frac{1}{i!}$ \\
1 & 2 & 1 & 2.0000000000000000000 \\ 
2 & 5 & 2 & 2.5000000000000000000 \\ 
3 & 8 & 3 & 2.6666666666666666667 \\ 
4 & 65 & 24 & 2.7083333333333333333 \\ 
5 & 163 & 60 & 2.7166666666666666667 \\ 
6 & 1957 & 720 & 2.7180555555555555556 \\ 
7 & 685 & 252 & 2.7182539682539682540 \\ 
8 & 109601 & 40320 & 2.7182787698412698413 \\ 
9 & 98641 & 36288 & 2.7182815255731922399 \\ 
10 & 9864101 & 3628800 & 2.7182818011463844797 \\ 
11 & 13563139 & 4989600 & 2.7182818261984928652 \\ 
12 & 260412269 & 95800320 & 2.7182818282861685639 \\ 
13 & 8463398743 & 3113510400 & 2.7182818284467590023 \\ 
14 & 47395032961 & 17435658240 & 2.7182818284582297479 \\ 
15 & 888656868019 & 326918592000 & 2.7182818284589944643 \\ 
16 & 56874039553217 & 20922789888000 & 2.7182818284590422591 \\ 
17 & 7437374403113 & 2736057139200 & 2.7182818284590450705 \\ 
18 & 17403456103284421 & 6402373705728000 & 2.7182818284590452267 \\ 
19 & 82666416490601 & 30411275102208 & 2.7182818284590452349 \\ 
20 & 6613313319248080001 & 2432902008176640000 & 2.7182818284590452353 \\ 
21 & 69439789852104840011 & 25545471085854720000 & 2.7182818284590452354 \\ 
22 & 611070150698522592097 & 224800145555521536000 & 2.7182818284590452354 \\ 
23 & 1351405140967886501753 & 497154168055480320000 & 2.7182818284590452354 \\ 
24 & 337310723185584470837549 & 124089680346647887872000 & 2.7182818284590452354 \\ 
25 & 85351903640077042215979 & 31399210614030336000000 & 2.7182818284590452354 \\
\end{tabular}

As the table shows, small values of $n$, are sufficient to give a good approximation of the natural log base $e$, which at $n = 21$ is at 20 places undistinguishable from $e$.
%%%%%
%%%%%
\end{document}
