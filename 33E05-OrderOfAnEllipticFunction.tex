\documentclass[12pt]{article}
\usepackage{pmmeta}
\pmcanonicalname{OrderOfAnEllipticFunction}
\pmcreated{2013-03-22 15:44:35}
\pmmodified{2013-03-22 15:44:35}
\pmowner{rspuzio}{6075}
\pmmodifier{rspuzio}{6075}
\pmtitle{order of an elliptic function}
\pmrecord{8}{37694}
\pmprivacy{1}
\pmauthor{rspuzio}{6075}
\pmtype{Definition}
\pmcomment{trigger rebuild}
\pmclassification{msc}{33E05}

% this is the default PlanetMath preamble.  as your knowledge
% of TeX increases, you will probably want to edit this, but
% it should be fine as is for beginners.

% almost certainly you want these
\usepackage{amssymb}
\usepackage{amsmath}
\usepackage{amsfonts}

% used for TeXing text within eps files
%\usepackage{psfrag}
% need this for including graphics (\includegraphics)
%\usepackage{graphicx}
% for neatly defining theorems and propositions
%\usepackage{amsthm}
% making logically defined graphics
%%%\usepackage{xypic}

% there are many more packages, add them here as you need them

% define commands here
\begin{document}
The \emph{order} of an elliptic function is the number of poles of the function contained within a fundamental period parallelogram, counted with multiplicity.
Sometimes the term ``degree'' is also used --- this usage agrees with the
theory of Riemann surfaces.

This order is always a finite number; this follows from the fact that a meromorphic function can only have a finite number of poles in a compact region (such as the closure of a period parallelogram).  As it turns out, the
order can be any integer greater than 1.
%%%%%
%%%%%
\end{document}
