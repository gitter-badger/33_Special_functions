\documentclass[12pt]{article}
\usepackage{pmmeta}
\pmcanonicalname{AlternateIntegralRepresentationOfBetaFunction2}
\pmcreated{2013-03-22 18:34:31}
\pmmodified{2013-03-22 18:34:31}
\pmowner{karstenb}{16623}
\pmmodifier{karstenb}{16623}
\pmtitle{alternate integral representation of beta function (2)}
\pmrecord{8}{41299}
\pmprivacy{1}
\pmauthor{karstenb}{16623}
\pmtype{Result}
\pmcomment{trigger rebuild}
\pmclassification{msc}{33B15}

\endmetadata

\usepackage{amssymb}
\usepackage{amsmath}
\usepackage{amsfonts}
\usepackage{amsthm}
\usepackage[sort&compress]{natbib}

%\usepackage{psfrag}
%\usepackage{graphicx}
%%%\usepackage{xypic}


% commands
\newcommand{\comment}[1]{\small{(\,\textit{#1}\;)}}


\begin{document}
Substitute $x := \frac{1}{1+s}$, $dx = \frac{-1}{(1 + s)^2}\,ds$:
\begin{align*}
\int_{0}^{1} x^{p-1} (1 - x)^{q-1} \,dx &= \int_{0}^{\infty} \frac{1}{(1+s)^{p+1}} \left(\frac{s}{1+s}\right)^{q-1}\,ds \\
&= \int_{0}^{\infty} \frac{s^{q-1}}{(1 + s)^{p+q}} \,ds
\end{align*}

Since $B(p,q) = B(q,p)$ this gives:
\begin{align*}
\int_{0}^{\infty} \frac{s^{p-1}}{(1 + s)^{p+q}} \,ds &= \frac{\Gamma(p) \Gamma(q)}{\Gamma(p+q)}
\end{align*}
%%%%%
%%%%%
\end{document}
