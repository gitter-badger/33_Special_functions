\documentclass[12pt]{article}
\usepackage{pmmeta}
\pmcanonicalname{MittagLefflerFunction}
\pmcreated{2013-03-22 14:54:34}
\pmmodified{2013-03-22 14:54:34}
\pmowner{rspuzio}{6075}
\pmmodifier{rspuzio}{6075}
\pmtitle{Mittag-Leffler function}
\pmrecord{5}{36594}
\pmprivacy{1}
\pmauthor{rspuzio}{6075}
\pmtype{Definition}
\pmcomment{trigger rebuild}
\pmclassification{msc}{33E12}

\endmetadata

% this is the default PlanetMath preamble.  as your knowledge
% of TeX increases, you will probably want to edit this, but
% it should be fine as is for beginners.

% almost certainly you want these
\usepackage{amssymb}
\usepackage{amsmath}
\usepackage{amsfonts}

% used for TeXing text within eps files
%\usepackage{psfrag}
% need this for including graphics (\includegraphics)
%\usepackage{graphicx}
% for neatly defining theorems and propositions
%\usepackage{amsthm}
% making logically defined graphics
%%%\usepackage{xypic}

% there are many more packages, add them here as you need them

% define commands here
\begin{document}
The Mittag-Leffler function $E_{\alpha \beta}$ is a complex function which depends on two complex parameters $\alpha$ and $\beta$.  It may be defined by the following series when the real part of $\alpha$ is strictly positive:
 $$E_{\alpha \beta} (z) = \sum_{k=0}^\infty {z^k \over \Gamma (\alpha k + \beta)}$$
In this case, the series converges for all values of the argument $z$, so the Mittag-Leffler function is an entire function.
%%%%%
%%%%%
\end{document}
