\documentclass[12pt]{article}
\usepackage{pmmeta}
\pmcanonicalname{OrthogonalityOfChebyshevPolynomialsFromRecursion}
\pmcreated{2013-03-22 18:54:46}
\pmmodified{2013-03-22 18:54:46}
\pmowner{rspuzio}{6075}
\pmmodifier{rspuzio}{6075}
\pmtitle{orthogonality of Chebyshev polynomials from recursion}
\pmrecord{6}{41762}
\pmprivacy{1}
\pmauthor{rspuzio}{6075}
\pmtype{Proof}
\pmcomment{trigger rebuild}
\pmclassification{msc}{33C45}
\pmclassification{msc}{33D45}
\pmclassification{msc}{42C05}

% this is the default PlanetMath preamble.  as your knowledge
% of TeX increases, you will probably want to edit this, but
% it should be fine as is for beginners.

% almost certainly you want these
\usepackage{amssymb}
\usepackage{amsmath}
\usepackage{amsfonts}

% used for TeXing text within eps files
%\usepackage{psfrag}
% need this for including graphics (\includegraphics)
%\usepackage{graphicx}
% for neatly defining theorems and propositions
%\usepackage{amsthm}
% making logically defined graphics
%%%\usepackage{xypic}

% there are many more packages, add them here as you need them

% define commands here

\begin{document}
In this entry, we shall demonstrate the orthogonality
relation of the Chebyshev polynomials from their
recursion relation.  Recall that this relation reads as
\[
 T_{n+1} (x) - 2 x T_n (x) + T_{n-1} = 0
\]
with initial conditions $T_0 (x) = 1$ and $T_1 (x) = x$.
The relation we seek to demonstrate is
\[
 \int_{-1}^{+1} dx \, 
   {T_m (x) T_n (x) \over \sqrt {1 - x^2}} = 0
\]
when $m \neq n$.

We start with the observation that $T_n$ is an even function
when $n$ is even and an odd function when $n$ is odd.  That
this is true for $T_0$ and $T_1$ follows immediately from their
definitions.  When $n > 1$, we may induce this from the 
recursion.  Suppose that $T_m (-x) = (-1)^m T_m (x)$ when 
$m < n$. Then we have
\begin{align*}
 T_{n+1} (-x) &=
  2 (-x) T_n (-x) - T_{n-1} (-x) \\ &=
  - (-1)^{n} 2 x T_n (x) - (-1)^{n-1} T_{n-1} (x) \\ &=
  (-1)^{n+1} (2 x T_n (x) - T_{n-1} (x)) \\ &=
  (-1)^{n+1} T_{n+1} (x) .
\end{align*}

From this observation, we may immediately conclude half
of orthogonality.  Suppose that $m$ and $n$ are nonnegative
integers whose difference is odd.  Then $T_m (-x) T_n (-x) 
= - T_m (x) T_n (x)$, so we have
\[
 \int_{-1}^{+1} dx \, 
   {T_m (x) T_n (x) \over \sqrt {1 - x^2}} = 0
\]
because the integrand is an odd function of $x$.

To cover the remaining cases, we shall proceed by induction.
Assume that $T_k$ is orthogonal to $T_m$ whenever $m \le n$
and $k \le n$ and $m \neq k$.  By the conclusions of last 
paragraph, we know that $T_{n+1}$ is orthogonal to $T_n$.  
Assume then that $m \le n-1$.  Using the recursion, we have
\begin{align*}
 \int_{-1}^{+1} dx \, 
  {T_m (x) T_{n+1} (x) \over \sqrt {1 - x^2}} &=
 2 \int_{-1}^{+1} dx \, 
  {x T_m (x) T_{n} (x) \over \sqrt {1 - x^2}} -
 \int_{-1}^{+1} dx \, 
  {T_m (x) T_{n-1} (x) \over \sqrt {1 - x^2}} \\ &=
 \int_{-1}^{+1} dx \, 
  {T_{m+1} (x) T_{n} (x) \over \sqrt {1 - x^2}} +
 \int_{-1}^{+1} dx \, 
  {T_{m-1} (x) T_{n} (x) \over \sqrt {1 - x^2}} -
 \int_{-1}^{+1} dx \, 
  {T_m (x) T_{n-1} (x) \over \sqrt {1 - x^2}}
\end{align*}
By our assumption, each of the three integrals is zero,
hence $T_{n+1}$ is orthogonal to $T_m$, so we conclude
that $T_k$ is orthogonal to $T_m$ when $m \le n+1$ and
$k \le n+1$ and $m \neq k$.
%%%%%
%%%%%
\end{document}
