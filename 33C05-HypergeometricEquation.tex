\documentclass[12pt]{article}
\usepackage{pmmeta}
\pmcanonicalname{HypergeometricEquation}
\pmcreated{2013-03-22 14:45:53}
\pmmodified{2013-03-22 14:45:53}
\pmowner{rspuzio}{6075}
\pmmodifier{rspuzio}{6075}
\pmtitle{hypergeometric equation}
\pmrecord{6}{36409}
\pmprivacy{1}
\pmauthor{rspuzio}{6075}
\pmtype{Definition}
\pmcomment{trigger rebuild}
\pmclassification{msc}{33C05}

\endmetadata

% this is the default PlanetMath preamble.  as your knowledge
% of TeX increases, you will probably want to edit this, but
% it should be fine as is for beginners.

% almost certainly you want these
\usepackage{amssymb}
\usepackage{amsmath}
\usepackage{amsfonts}

% used for TeXing text within eps files
%\usepackage{psfrag}
% need this for including graphics (\includegraphics)
%\usepackage{graphicx}
% for neatly defining theorems and propositions
%\usepackage{amsthm}
% making logically defined graphics
%%%\usepackage{xypic}

% there are many more packages, add them here as you need them

% define commands here
\begin{document}
The hypergeometric equation is the following linear ordinary differential equation:
 $$x (1 - x) y'' + (c - (a + b + 1) x ) y' - aby = 0$$
(Here, $a$, $b$, and $c$ are complex constants.)

The solutions of this equation may be expressed in terms of the hypergeometric function, hence the name.

The hypergeometric equation is a Fuchsian differential equation with singularities at $0$, $1$, and $\infty$.  By a suitable change of variables, any second order Fuchsian differential equation may be converted into a hypergeometric equation.
%%%%%
%%%%%
\end{document}
