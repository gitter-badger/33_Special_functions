\documentclass[12pt]{article}
\usepackage{pmmeta}
\pmcanonicalname{DualityOfGudermannianAndItsInverseFunction}
\pmcreated{2013-03-22 19:06:41}
\pmmodified{2013-03-22 19:06:41}
\pmowner{pahio}{2872}
\pmmodifier{pahio}{2872}
\pmtitle{duality of Gudermannian and its inverse function}
\pmrecord{7}{42004}
\pmprivacy{1}
\pmauthor{pahio}{2872}
\pmtype{Topic}
\pmcomment{trigger rebuild}
\pmclassification{msc}{33B10}
\pmclassification{msc}{26E05}
\pmclassification{msc}{26A09}
\pmclassification{msc}{26A48}
\pmrelated{InverseGudermannianFunction}
\pmrelated{IdealInvertingInPruferRing}

\endmetadata

% this is the default PlanetMath preamble.  as your knowledge
% of TeX increases, you will probably want to edit this, but
% it should be fine as is for beginners.

% almost certainly you want these
\usepackage{amssymb}
\usepackage{amsmath}
\usepackage{amsfonts}

% used for TeXing text within eps files
%\usepackage{psfrag}
% need this for including graphics (\includegraphics)
%\usepackage{graphicx}
% for neatly defining theorems and propositions
 \usepackage{amsthm}
% making logically defined graphics
%%%\usepackage{xypic}

% there are many more packages, add them here as you need them

% define commands here

\theoremstyle{definition}
\newtheorem*{thmplain}{Theorem}

\begin{document}
\PMlinkescapeword{formula}

There are a lot of formulae concerning the Gudermannian function and its inverse function containing a hyperbolic function or a trigonometric function or both, such that if we change functions of one kind to the corresponding functions of the other kind, then the new formula also is true.

Some exemples:

\begin{align}
\mbox{gd}\,x \;=\; \int_0^x\!\frac{dt}{\cosh{t}}, \qquad \mbox{gd}^{-1}x \;=\; \int_0^x\!\frac{dt}{\cos{t}}
\end{align}

\begin{align}
\frac{d}{dx}\mbox{gd}\,x \;=\; \frac{1}{\cosh{x}}, \qquad \frac{d}{dx}\mbox{gd}^{-1}\,x \;=\; \frac{1}{\cos{x}}
\end{align}

\begin{align}
\tan(\mbox{gd}\,x) \;=\; \sinh{x}, \qquad \tanh(\mbox{gd}^{-1}x) \;=\; \sin{x}
\end{align}

\begin{align}
\sin(\mbox{gd}\,x) \;=\; \tanh{x}, \qquad \sinh(\mbox{gd}^{-1}x) \;=\; \tan{x}
\end{align}

\begin{align}
\tan\frac{\mbox{gd}\,x}{2} \;=\; \tanh\frac{x}{2}, \qquad \tanh\frac{\mbox{gd}^{-1}x}{2} \;=\; \tan\frac{x}{2}
\end{align}



For proving (5) we can check that
$$\frac{d}{dx}[2\arctan(\tanh\frac{x}{2})] \;=\; \frac{1}{\cosh{x}},$$
and since both the expression in the brackets and the \PMlinkid{Gudermannian}{11997} vanish in the origin, we have
$$\mbox{gd}\,x \;\equiv\; 2\arctan(\tanh\frac{x}{2}).$$
This equation implies (5).\\


The \PMlinkname{duality}{DualityInMathematics} of the formula pairs may be explained by the equality
\begin{align}
\mbox{gd}\,ix \;=\; i\,\mbox{gd}^{-1}x.
\end{align}
%%%%%
%%%%%
\end{document}
