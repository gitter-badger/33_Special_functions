\documentclass[12pt]{article}
\usepackage{pmmeta}
\pmcanonicalname{IntegralRepresentationOfTheHypergeometricFunction}
\pmcreated{2013-03-22 14:35:14}
\pmmodified{2013-03-22 14:35:14}
\pmowner{rspuzio}{6075}
\pmmodifier{rspuzio}{6075}
\pmtitle{integral representation of the hypergeometric function}
\pmrecord{6}{36150}
\pmprivacy{1}
\pmauthor{rspuzio}{6075}
\pmtype{Theorem}
\pmcomment{trigger rebuild}
\pmclassification{msc}{33C05}

\endmetadata

% this is the default PlanetMath preamble.  as your knowledge
% of TeX increases, you will probably want to edit this, but
% it should be fine as is for beginners.

% almost certainly you want these
\usepackage{amssymb}
\usepackage{amsmath}
\usepackage{amsfonts}

% used for TeXing text within eps files
%\usepackage{psfrag}
% need this for including graphics (\includegraphics)
%\usepackage{graphicx}
% for neatly defining theorems and propositions
%\usepackage{amsthm}
% making logically defined graphics
%%%\usepackage{xypic}

% there are many more packages, add them here as you need them

% define commands here
\begin{document}
When $\Re c > \Re b > 0$, one has the representation
 $$F(a,b;c;z) = {\Gamma (c) \over \Gamma (b) \Gamma (c-b)} \int_0^1 t^{b-1} (1-t)^{c-b-1}  (1-tz)^{-a} \, dt$$

Note that the conditions on $b$ and $c$ are necessary for the integral to be convergent at the endpoints $0$ and $1$.  To see that this integral indeed equals the hypergeometric function, it suffices to consider the case $|z| < 1$ since both sides of the equation are analytic functions of $z$.  (This follows from the rigidity theorem for analytic functions although some care is required because the function is multiply-valued.)  With this assumption, $|tz| < 1$ if $t$ is a real number in the interval $[0,1]$ and hence, $(1 - tz)^{-a}$ may be expanded in a power series.  Substituting this series in the right hand side of the formula above gives
 $${\Gamma (c) \over \Gamma (b) \Gamma (c-b)} \int_0^1 \sum_{k=0}^\infty t^{b-1} (1-t)^{c-b-1} {\Gamma (k-a+1) \over \Gamma (1-a) \Gamma (k+1)} (-tz)^k  \, dt$$
Since the series is uniformly convergent, it is permissible to integrate term-by-term.  Interchanging integration and summation and pulling constants outside the integral sign, one obtains
 $${\Gamma (c) \over \Gamma (b) \Gamma (c-b)} \sum_{k=0}^\infty {\Gamma (k-a+1) \over \Gamma (1-a) \Gamma (k+1)} (-z)^k \int_0^1 (1-t)^{c-b-1} t^{b+k-1} dt$$
The integrals appearing inside the sum are Euler beta functions.  Expressing them in terms of gamma functions and simplifying, one sees that this integral indeed equals the hypergeometric function.

The hypergeometic function is multiply-valued.  To obtain different branches of the hypergeometric function, one can vary the path of integration.
%%%%%
%%%%%
\end{document}
