\documentclass[12pt]{article}
\usepackage{pmmeta}
\pmcanonicalname{ProofOfJacobisIdentityForvarthetaFunctions}
\pmcreated{2013-03-22 14:47:01}
\pmmodified{2013-03-22 14:47:01}
\pmowner{rspuzio}{6075}
\pmmodifier{rspuzio}{6075}
\pmtitle{proof of Jacobi's identity for $\vartheta$ functions}
\pmrecord{19}{36434}
\pmprivacy{1}
\pmauthor{rspuzio}{6075}
\pmtype{Proof}
\pmcomment{trigger rebuild}
\pmclassification{msc}{33E05}

% this is the default PlanetMath preamble.  as your knowledge
% of TeX increases, you will probably want to edit this, but
% it should be fine as is for beginners.

% almost certainly you want these
\usepackage{amssymb}
\usepackage{amsmath}
\usepackage{amsfonts}

% used for TeXing text within eps files
%\usepackage{psfrag}
% need this for including graphics (\includegraphics)
%\usepackage{graphicx}
% for neatly defining theorems and propositions
%\usepackage{amsthm}
% making logically defined graphics
%%%\usepackage{xypic}

% there are many more packages, add them here as you need them

% define commands here
\begin{document}
We start with the Fourier transform of $f(x) = e^{i \pi \tau x^2 + 2 i x z}$:
 $$\int_{-\infty}^{+\infty} e^{i \pi \tau x^2 + 2 i x z} e^{2 \pi ixy} \, dx = (- i \tau)^{-1/2} e^{-i {(z + \pi y)^2 \over \pi \tau}}$$

Applying the Poisson summation formula, we obtain the following:
 $$\sum_{n=-\infty}^{+\infty} e^{i \pi \tau n^2 + 2 i n z} = (- i \tau)^{-1/2} \sum_{n=-\infty}^{+\infty} e^{-i {(z + \pi n)^2 \over \pi \tau}}$$

The left hand \PMlinkescapetext{side} equals $\vartheta_3 (z \mid \tau)$.  The right hand \PMlinkescapetext{side} can be rewritten as follows:
 $$\sum_{n=-\infty}^{+\infty} e^{-i {(z + \pi n)^2 \over \pi \tau}} = e^{-i {z^2 \over \pi \tau}} \sum_{n=-\infty}^{+\infty} e^{-i {\pi n^2 \over \tau} - {2 i n z \over \tau}} = e^{-i {z^2 \over \pi \tau}} \vartheta_3 (z / \tau  \mid -1 / \tau)$$

Combining the two expressions yields
 $$\vartheta_3 (z \mid \tau) = e^{-i {z^2 \over \pi \tau}} \vartheta_3 (z / \tau  \mid -1 / \tau)$$
%%%%%
%%%%%
\end{document}
