\documentclass[12pt]{article}
\usepackage{pmmeta}
\pmcanonicalname{NaturalLogarithm}
\pmcreated{2013-03-22 12:28:28}
\pmmodified{2013-03-22 12:28:28}
\pmowner{mathwizard}{128}
\pmmodifier{mathwizard}{128}
\pmtitle{natural logarithm}
\pmrecord{13}{32666}
\pmprivacy{1}
\pmauthor{mathwizard}{128}
\pmtype{Definition}
\pmcomment{trigger rebuild}
\pmclassification{msc}{33B10}
\pmrelated{MatrixLogarithm}
\pmrelated{ComplexLogarithm}

\endmetadata

% this is the default PlanetMath preamble.  as your knowledge
% of TeX increases, you will probably want to edit this, but
% it should be fine as is for beginners.

% almost certainly you want these
\usepackage{amssymb}
\usepackage{amsmath}
\usepackage{amsfonts}

% used for TeXing text within eps files
%\usepackage{psfrag}
% need this for including graphics (\includegraphics)
\usepackage{graphicx}
% for neatly defining theorems and propositions
%\usepackage{amsthm}
% making logically defined graphics
%%%\usepackage{xypic} 

% there are many more packages, add them here as you need them

% define commands here
\begin{document}
The natural logarithm of a number is the logarithm in base of Euler's number $e$. It can be defined as the map $\ln\colon\mathbb{R}_+\to\mathbb{R}$ satisfying
\begin{equation}\label{eq:definition}
\ln(x)\colon =\int_1^x\frac{1}{t}dt.
\end{equation}
Figure \ref{fig:graph} shows the graph of $\ln$.
\begin{figure}[htbp]
\begin{centering}
\includegraphics[angle=270,scale=0.5]{logarithm.ps}
\caption{The graph of $\ln(x)$.}\label{fig:graph}
\end{centering}
\end{figure}
Instead of $\ln$ many mathematicians write $\log$, physicists (and calculators) however consider $\log$ as the symbol for the logarithm in base 10.
One can show that the function defined in this way is the inverse of the exponential function. Indeed with equation (\ref{eq:definition}) we have
$$\frac{d}{dx}\ln(e^x)=e^x\cdot\frac{1}{e^x}=1,$$
so there exists $C\in\mathbb{R}$ such that
$$\ln(e^x)=x+C.$$
Since $\ln(e^0)=0$ we have $C=0$.
One can also prove that the above integral has the defining properties of a logarithm. For example if $x,y\in\mathbb{R}_+$ we have
\begin{eqnarray*}
\ln(xy)&=&\int_1^x\frac{1}{t}\mathit{dt}+\int_x^{xy}\frac{1}{t}\mathit{dt}\\
&=&\ln(x)+\int_x^{xy}\frac{1}{t}dt.
\end{eqnarray*}
Now applying the substitution law with $u\colon =\frac{t}{x}$ we have
$$\int_x^{xy}\frac{1}{t}\mathit{dt}=\int_1^y\frac{1}{u}\mathit{du}=\ln(y),$$
so we have
$$\ln(xy)=\ln(x)+\ln(y).$$
The natural logarithm can also be represented as a power series around $1$. For $-1<x\le 1$ we have
\begin{equation}\label{eq:power}
\ln(1+x)=\sum_{k=1}^\infty \frac{(-1)^{k+1}}{k}x^k.
\end{equation}
This series is divergent at $x=-1$ but for $x=1$ we have convergence due to Leibniz's theorem and obtain
$$\ln(2)=\sum_{k=1}^\infty \frac{(-1)^{k+1}}{k}.$$
In real analysis there is no reasonable way to extend the logarithm to negative numbers. In complex analysis the situation is a bit more complicated. Basically one can use the Euler relation to write a non-zero complex number $z$ in the form $z=Re^{i\varphi}$ with $R,\varphi\in\mathbb{R}$. 
We could try to define the complex logarithm $\ln(z)$ to be $\ln(R)+i\varphi$. However $\varphi$ is unique only up to addition of a multiple of $2\pi$.
While at first glance this does not appear to be very problematic, it actually prevents one from setting up a continuous logarithm on the complex plane (without 0, where the logarithm should be infinite). Say for example that we let the imaginary part of our logarithm take values from $-\pi$ to $\pi$. Then
$$\lim_{\varphi\searrow-\pi}\ln e^{i\varphi}=-i\pi$$
and
$$\lim_{\varphi\nearrow\pi}\ln e^{i\varphi}=i\pi$$
while $e^{i\varphi}\to-1$ for both limits. Therefore the logarithm we defined is not continuous at -1. The same argument allows one to show that it is not continuous on the negative real numbers. In fact you can only define a continuous complex logarithm on a sliced plane, i.e. the complex plane with a half-line starting at 0 removed. 
%%%%%
%%%%%
\end{document}
