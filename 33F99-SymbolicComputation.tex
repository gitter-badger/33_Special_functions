\documentclass[12pt]{article}
\usepackage{pmmeta}
\pmcanonicalname{SymbolicComputation}
\pmcreated{2013-03-22 12:04:20}
\pmmodified{2013-03-22 12:04:20}
\pmowner{akrowne}{2}
\pmmodifier{akrowne}{2}
\pmtitle{symbolic computation}
\pmrecord{18}{31139}
\pmprivacy{1}
\pmauthor{akrowne}{2}
\pmtype{Topic}
\pmcomment{trigger rebuild}
\pmclassification{msc}{33F99}
\pmclassification{msc}{17-08}
\pmclassification{msc}{16Z05}
\pmclassification{msc}{13P99}
\pmclassification{msc}{12Y05}
\pmclassification{msc}{11Y40}
\pmclassification{msc}{14Q99}
\pmclassification{msc}{68W30}
\pmsynonym{formula manipulation}{SymbolicComputation}
\pmsynonym{algebraic computation}{SymbolicComputation}
\pmdefines{CAS}
\pmdefines{computer algebra systems}

\endmetadata

\usepackage{amssymb}
\usepackage{amsmath}
\usepackage{amsfonts}
%\usepackage{graphicx}
%%%%\usepackage{xypic}
\begin{document}
\section{Symbolic Computation}

Also called \emph{formula manipulation} or \emph{algebraic computation}.  

\emph{Symbolic computation} refers to the automatic transformation of mathematical expressions in symbolic form, hence in an exact way, as opposed to numerical and hence limited-precision floating-point computation.  Typical operations include differentiation and integration, linear algebra and matrix calculus, operations with polynomials, or the simplification of algebraic expressions.

Programs or systems in this area which provide a language interface are called \emph{Computer Algebra Systems} (or CASes).  There are also symbolic computation libraries for existing programming languages. 

Primarily designed for applications in theoretical physics or mathematics, these systems (which are often interactive in the case of CASes) can be used in any area where straightforward but tedious or lengthy calculations with formulae are required. 

\section{Systems}

Some well known, general symbolic computation CASes are:

\begin{itemize}

\item \PMlinkexternal{Axiom}{http://www.axiom-developer.org}
\item \PMlinkexternal{Macsyma}{http://www.scientek.com/macsyma/mxmain.htm}
\item \PMlinkexternal{GNU Maxima}{http://maxima.sourceforge.net/}
\item \PMlinkexternal{Maple}{http://www.maplesoft.com/products/maple/}
\item \PMlinkexternal{Mathematica}{http://www.wolfram.com/products/mathematica/index.html}
\item \PMlinkexternal{Reduce}{http://www.uni-koeln.de/REDUCE/}

\end{itemize}

These systems have different scope and facilities, and some are easier to use or to access than others.  There is a trend away from generalized CAS systems to more specialized, application-specific systems, such as:

\begin{itemize}

\item \PMlinkexternal{SINGULAR}{http://www.singular.uni-kl.de/} (algebraic geometry, esp. singular varieties)
\item \PMlinkexternal{PARI-GP}{http://pari.math.u-bordeaux.fr/} (computations on curves)
\item \PMlinkexternal{GAP}{http://www-gap.dcs.st-and.ac.uk/~gap/} (group theory)

\end{itemize}

Some non-CAS symbolic computation libraries, with their supported languages, are:

\begin{itemize}
\item \PMlinkexternal{GiNaC}{http://www.ginac.de/} (C++)
\end{itemize}

\begin{thebibliography}{3}

\bibitem{DABB} Based on content from the \PMlinkexternal{Data Analysis Briefbook}{http://rkb.home.cern.ch/rkb/titleA.html}

\end{thebibliography}
%%%%%
%%%%%
%%%%%
\end{document}
