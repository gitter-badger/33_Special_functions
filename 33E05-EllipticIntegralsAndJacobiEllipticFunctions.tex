\documentclass[12pt]{article}
\usepackage{pmmeta}
\pmcanonicalname{EllipticIntegralsAndJacobiEllipticFunctions}
\pmcreated{2013-03-22 13:58:28}
\pmmodified{2013-03-22 13:58:28}
\pmowner{mathcam}{2727}
\pmmodifier{mathcam}{2727}
\pmtitle{elliptic integrals and Jacobi elliptic functions}
\pmrecord{7}{34746}
\pmprivacy{1}
\pmauthor{mathcam}{2727}
\pmtype{Definition}
\pmcomment{trigger rebuild}
\pmclassification{msc}{33E05}
\pmrelated{ArithmeticGeometricMean}
\pmrelated{PerimeterOfEllipse}
\pmdefines{elliptic integral}
\pmdefines{Jacobi elliptic function}
\pmdefines{Jacobian elliptic function}
\pmdefines{complementary modulus}
\pmdefines{complete elliptic integral}

\endmetadata

\usepackage{amssymb}
\usepackage{amsmath}
\usepackage{amsfonts}
\begin{document}
\PMlinkescapeword{periods} \PMlinkescapeword{simple}
\PMlinkescapeword{names} \PMlinkescapeword{terms}
\PMlinkescapeword{Jacobian} \PMlinkescapeword{complex}
\PMlinkescapeword{complete} \PMlinkescapeword{incomplete}
\textbf{Elliptic integrals}

For a modulus $0<k<1$ (while here, we define the \emph{complementary modulus} to $k$ to be the positive number $k'$ with $k^2+k'^2=1$) , write
\begin{eqnarray}
F(\phi,k)&=&\int_0^\phi\frac{d\theta}{\sqrt{1-k^2\sin^2\theta}} \\
E(\phi,k)&=&\int_0^\phi\sqrt{1-k^2\sin^2\theta}\,d\theta \\
\Pi(n,\phi,k)&=&\int_0^\phi\frac{d\theta}{(1+n\sin^2\theta)\sqrt{1-k^2\sin^2\theta}}
\end{eqnarray}
The change of variable $x=\sin\phi$ turns these into
\begin{eqnarray}
F_1(x,k)&=&\int_0^x\frac{dv}{\sqrt{(1-v^2)(1-k^2v^2)}} \\
E_1(x,k)&=&\int_0^x\sqrt{\frac{1-k^2v^2}{1-v^2}}\,dv \\
\Pi_1(n,x,k)&=&\int_0^x\frac{dv}{(1+nv^2)\sqrt{(1-v^2)(1-k^2v^2)}}
\end{eqnarray}
The first three functions are known as Legendre's form of the incomplete
elliptic integrals of the first, second, and third kinds respectively.
Notice that (2) is the special case $n=0$ of (3).
The latter three are known as Jacobi's form of those integrals.
If $\phi=\pi/2$, or $x=1$, they are called complete rather than incomplete
integrals, and we refer to the auxiliary elliptic integrals $K(k)=F(\pi/2,k)$, $E(k)=E(\pi/2,k)$, etc.

One use for elliptic integrals is to systematize the evaluation of
certain other integrals.
In particular, let $p$ be a third- or fourth-degree polynomial
in one variable, and let $y=\sqrt{p(x)}$.
If $q$ and $r$ are any two polynomials in two variables, then the
indefinite integral
$$\int\frac{q(x,y)}{r(x,y)}\,dx$$
has a ``closed form'' in terms of the above incomplete elliptic integrals,
together with elementary functions and their inverses.

\textbf{Jacobi's elliptic functions}

In (1) we may regard $\phi$ as a function of $F$, or vice versa.
The notation used is
$$\phi=\mathrm{am}\,u\qquad u=\mathrm{arg}\,\phi$$
and $\phi$ and $u$ are known as the amplitude and argument respectively.
But $x=\sin\phi=\sin\mathrm{am}\,u$.
The function $u\mapsto \sin\mathrm{am}\,u=x$
is denoted by $\mathrm{sn}$ and is one of four \emph{Jacobi (or Jacobian)
elliptic functions}. The four are:
\begin{eqnarray*}
\mathrm{sn}\,u&=&x \\
\mathrm{cn}\,u&=&\sqrt{1-x^2} \\
\mathrm{tn}\,u&=&\frac{\mathrm{sn}\,u}{\mathrm{cn}\,u} \\
\mathrm{dn}\,u&=&\sqrt{1-k^2x^2}
\end{eqnarray*}

When the Jacobian elliptic functions are extended to complex arguments,
they are doubly periodic and have two poles in any parallelogram of
periods; both poles are simple.
%%%%%
%%%%%
\end{document}
