\documentclass[12pt]{article}
\usepackage{pmmeta}
\pmcanonicalname{ValuesOfComplexCosine}
\pmcreated{2013-03-22 17:36:29}
\pmmodified{2013-03-22 17:36:29}
\pmowner{pahio}{2872}
\pmmodifier{pahio}{2872}
\pmtitle{values of complex cosine}
\pmrecord{16}{40024}
\pmprivacy{1}
\pmauthor{pahio}{2872}
\pmtype{Topic}
\pmcomment{trigger rebuild}
\pmclassification{msc}{33B10}
\pmclassification{msc}{30A99}
%\pmkeywords{cosine function}
%\pmkeywords{zeros of cosine}
\pmrelated{RealPart}
\pmrelated{PropertiesOfQuadraticEquation}
\pmrelated{TakingSquareRootAlgebraically}
\pmrelated{ComplexLogarithm}
\pmdefines{period strip}

% this is the default PlanetMath preamble.  as your knowledge
% of TeX increases, you will probably want to edit this, but
% it should be fine as is for beginners.

% almost certainly you want these
\usepackage{amssymb}
\usepackage{amsmath}
\usepackage{amsfonts}

% used for TeXing text within eps files
%\usepackage{psfrag}
% need this for including graphics (\includegraphics)
%\usepackage{graphicx}
% for neatly defining theorems and propositions
 \usepackage{amsthm}
% making logically defined graphics
%%%\usepackage{xypic}

% there are many more packages, add them here as you need them

% define commands here

\theoremstyle{definition}
\newtheorem*{thmplain}{Theorem}

\begin{document}
\PMlinkescapeword{root} \PMlinkescapeword{roots}

Since the complex cosine function \,$z \mapsto \cos{z}$\, has the prime period $2\pi$, the cosine attains all of its possible values in one of its {\em period strips}, for example in the period strip
\begin{align}
   \{z \in \mathbb{C}\,\vdots\;\, -\pi \;\leqq\; \mbox{Re}(z) \;<\; \pi\}.
\end{align}
For finding out which values the cosine function can attain in a period strip, we solve the equation \,$\cos{z} = w$,\, where $w$ is any complex number.  Using \PMlinkname{Euler's formula}{ComplexSineAndCosine}
     $$\cos{z} \;=\; \frac{e^{iz}+e^{-iz}}{2},$$
the equation may be written as
\begin{align}
(e^{iz})^2-2we^{iz}+1 \;=\; 0.
\end{align}
This is a quadratic equation in $e^{iz}$, whence we obtain the two \PMlinkname{roots}{Equation}
       $$e^{iz} \;=\; w\pm\sqrt{w^2-1}.$$
The product of the roots is 1, and therefore the roots are distinct from zero for all values of $w$.  If we set
    $$w+\sqrt{w^2-1} \;=\; re^{i\varphi} \quad (-\pi \;\leqq\; \varphi \;<\; \pi),$$
the other root is the inverse number
     $$w-\sqrt{w^2-1} \;=\; \frac{1}{r}e^{-i\varphi}.$$
The solution of the equation
       $$e^{iz} \;=\; re^{i\varphi}$$
is then obtained by taking the complex logarithm
    $$z \;=\; z_1 \;=\; \frac{1}{i}\log(re^{i\varphi}) \;=\; \varphi-i\ln{r}+n\cdot2\pi \quad (n\;\in\;\mathbb{Z}),$$
and the other solution of (2) is
     $$z \;=\; z_2 \;=\; -\varphi+i\ln{r}+n\cdot2\pi \quad (n\;\in\;\mathbb{Z}).$$
In the period strip (1), we have one solution $z_1$ and one solution $z_2$, both obtained with the value\, $n = 0$\, (except $z_2$ in the case\, $\varphi = -\pi$\, with\, $n = -1$).\, In (1), the points $z_1$ and $z_2$ are situated symmetrically with respect the origin.\, In the cases\, $w = 1$\, and\, $w = -1$,\, the equation (2) has double roots\, $z = 0$\, and\, $z = -\pi$,\, respectively; then we may say that $z_1$ and $z_2$ coincide.  Anyhow, we have the

\textbf{Theorem.}  In every period strip, cosine attains any complex value at two points.\\

\textbf{Example.}  The solution of the equation\, $\cos{z} = 2$\, is obtained from\, $e^{iz} = 2\!\pm\!\sqrt{3}$.  In the period strip (1) we get
 $$z \;=\; \frac{1}{i}\log(2\!\pm\!\sqrt{3}) \;=\; -i\ln(2\!\pm\!\sqrt{3})+0\cdot2\pi.$$
Since\, $2\!\pm\!\sqrt{3}$\, are inverse numbers of each other, we have as result the purely imaginary numbers 
\,$z = \pm{i}\ln(2\!+\!\sqrt{3})$.\\

From trigonometry, we know that the real zeros of cosine are the odd multiples of $\displaystyle\frac{\pi}{2}$; from these points, $\displaystyle\pm\frac{\pi}{2}$ belong to the period strip (1).  Thus $\displaystyle\pm\frac{\pi}{2}$ are the only points of (1) where the cosine vanishes.  Therefore, according to the preceding theorem, the well-known points
$$(2n\!+\!1)\frac{\pi}{2} \quad (n \;=\; 0,\,\pm1,\,\pm2,\,\ldots)$$
are the only zeros of the cosine function on the whole complex plane.\\

The values of complex cosine function may be transferred to the complex sine function by means of the complement formula 
$$\sin{z} \;=\; \cos(\frac{\pi}{2}-z).$$
One can think all points of the $z$-plane to bear the corresponding value of cosine, and then one can translate the plane in the direction of the real axis the distance $\displaystyle\frac{\pi}{2}$; then the values of the sine have been placed to their correct \PMlinkescapetext{places}.  So one has transferred also the above properties of cosine to sine.


\begin{thebibliography}{8}
\bibitem{L}{\sc Ernst Lindel\"of}: {\em Johdatus funktioteoriaan}.  Second edition. Mercatorin Kirjapaino Osakeyhti\"o, Helsinki (1936).
\bibitem{NP}{\sc R. Nevanlinna \& V. Paatero}: {\em Funktioteoria}.  Kustannusosakeyhti\"o Otava, Helsinki (1963).
\end{thebibliography} 



 



%%%%%
%%%%%
\end{document}
