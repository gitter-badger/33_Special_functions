\documentclass[12pt]{article}
\usepackage{pmmeta}
\pmcanonicalname{ExamplesOfEllipticFunctions}
\pmcreated{2013-03-22 13:54:04}
\pmmodified{2013-03-22 13:54:04}
\pmowner{alozano}{2414}
\pmmodifier{alozano}{2414}
\pmtitle{examples of elliptic functions}
\pmrecord{7}{34649}
\pmprivacy{1}
\pmauthor{alozano}{2414}
\pmtype{Example}
\pmcomment{trigger rebuild}
\pmclassification{msc}{33E05}
\pmrelated{EllipticFunction}
\pmrelated{WeierstrassWpFunction}
\pmdefines{Eisenstein series}

\endmetadata

% this is the default PlanetMath preamble.  as your knowledge
% of TeX increases, you will probably want to edit this, but
% it should be fine as is for beginners.

% almost certainly you want these
\usepackage{amssymb}
\usepackage{amsmath}
\usepackage{amsthm}
\usepackage{amsfonts}

% used for TeXing text within eps files
%\usepackage{psfrag}
% need this for including graphics (\includegraphics)
%\usepackage{graphicx}
% for neatly defining theorems and propositions
%\usepackage{amsthm}
% making logically defined graphics
%%%\usepackage{xypic}

% there are many more packages, add them here as you need them

% define commands here

\newtheorem{thm}{Theorem}
\newtheorem{defn}{Definition}
\newtheorem{prop}{Proposition}
\newtheorem{lemma}{Lemma}
\newtheorem{cor}{Corollary}

% Some sets
\newcommand{\Nats}{\mathbb{N}}
\newcommand{\Ints}{\mathbb{Z}}
\newcommand{\Reals}{\mathbb{R}}
\newcommand{\Complex}{\mathbb{C}}
\newcommand{\Rats}{\mathbb{Q}}
\begin{document}
{\bf Examples of Elliptic Functions}

Let $\Lambda \subset \Complex$ be a lattice generated by
$w_1,w_2$. Let $\Lambda^{\ast}$ denote $\Lambda-\{ 0 \}$.
\begin{enumerate}
\item The \emph{Weierstrass} $\wp$-function is defined by the
series
$$\wp(z;\Lambda)=\frac{1}{z^2}+\sum_{w\in\Lambda^{\ast}}\frac{1}{(z-w)^2}-\frac{1}{w^2}$$

\item The derivative of the Weierstrass $\wp$-function is also an
elliptic function
$$\wp'(z;\Lambda)=-2\sum_{w\in\Lambda^{\ast}}\frac{1}{(z-w)^3}$$

\item The \emph{Eisenstein series of weight} $2k$ for $\Lambda$ is
the series
$$\mathcal{G}_{2k}(\Lambda)=\sum_{w\in\Lambda^{\ast}}w^{-2k}$$
The Eisenstein series of weight $4$ and $6$ are of special
relevance in the theory of elliptic curves. In particular, the quantities $g_2$ and $g_3$ are usually defined as follows:
$$g_2=60\cdot\mathcal{G}_4(\Lambda),\quad
g_3=140\cdot\mathcal{G}_6(\Lambda)$$
\end{enumerate}

{\bf Remark:} The elliptic functions $\wp$, $\wp'$ and $\mathcal{G}_{2k}$ are related by the following important equation: 
\[\left( \wp'(z;\Lambda) \right)^2 = 4 \wp(z;\Lambda)^3 -
g_2(\Lambda) \wp(z;\Lambda) - g_3(\Lambda)\]
In particular, the previous equation provides an isomorphism between $\Complex/\Lambda$ and the elliptic curve $E : y^2=4x^3-g_2x-g_3$ given by:
$$\Complex/\Lambda \to  E, \quad z \mapsto (\wp(z;\Lambda),\wp'(z;\Lambda)).$$
%%%%%
%%%%%
\end{document}
