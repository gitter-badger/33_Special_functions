\documentclass[12pt]{article}
\usepackage{pmmeta}
\pmcanonicalname{ConvergenceOfArithmeticgeometricMean}
\pmcreated{2013-03-22 17:09:46}
\pmmodified{2013-03-22 17:09:46}
\pmowner{rspuzio}{6075}
\pmmodifier{rspuzio}{6075}
\pmtitle{convergence of arithmetic-geometric mean}
\pmrecord{13}{39474}
\pmprivacy{1}
\pmauthor{rspuzio}{6075}
\pmtype{Theorem}
\pmcomment{trigger rebuild}
\pmclassification{msc}{33E05}
\pmclassification{msc}{26E60}

% this is the default PlanetMath preamble.  as your knowledge
% of TeX increases, you will probably want to edit this, but
% it should be fine as is for beginners.

% almost certainly you want these
\usepackage{amssymb}
\usepackage{amsmath}
\usepackage{amsfonts}

% used for TeXing text within eps files
%\usepackage{psfrag}
% need this for including graphics (\includegraphics)
%\usepackage{graphicx}
% for neatly defining theorems and propositions
%\usepackage{amsthm}
% making logically defined graphics
%%%\usepackage{xypic}

% there are many more packages, add them here as you need them

% define commands here

\begin{document}
In this entry, we show that the arithmetic-geometric mean converges.
By the arithmetic-geometric means inequality, we know that the sequences 
of arithmetic and geometric means are both monotonic and bounded, so
they converge individually.  What still needs to be shown is that they 
converge to the same limit.

Define $x_n = a_n / g_n$.  By the arithmetic-geometric inequality, we
have $x_n \ge 1$.  By the defining recursions, we have
\[
x_{n+1} =
{a_{n+1} \over g_{n+1}} =
{a_n + g_n \over 2 \sqrt{a_n g_n}} =
{1 \over 2}
\left(
\sqrt{a_n \over g_n} + \sqrt{g_n \over a_n}
\right) =
{1 \over 2}
\left(
\sqrt{x_n} + 
{1 \over \sqrt{x_n}}
\right)
\]
Since $x_n \ge 1$, we have $1 / \sqrt{x_n} \le 1$, and $\sqrt{x_n} \le x_n$, hence
\[
x_{n+1} - 1 =
{1 \over 2}
\left(
\sqrt{x_n} + 
{1 \over \sqrt{x_n}}
-2
\right) \le
{1 \over 2}
({x_n} + 1-2) \le
{1 \over 2}
(x_n - 1) .
\]
From this inequality
\[
0 \le x_{n+1} - 1 \le
{1 \over 2} (x_n - 1) ,
\]
we may conclude that $x_n \to 1$ as $n \to \infty$, which , by the definition of $x_n$,
is equivalent to
\[
\lim_{n \to \infty} g_n =
\lim_{n \to \infty} a_n .
\]

Not only have we proven that the arithmetic-geometric mean converges, but we can infer
a rate of convergence from our proof.  Namely, we have that $0 \le x_n - 1 \le (x_0 - 1) 
/ 2^n$.  Hence, we see that the rate of convergence of $a_n$ and $g_n$ to the answer goes
as $O(2^{-n})$.

By more carefully bounding the recursion for $x_n$ above, we may obtain better estimates
of the rate of convergence.  We will now derive an inequality. Suppose that $y \ge 0$.
\begin{align*}
0 &\le y^5 + y^4 + 4 y^3 + 3 y^2 \\
y^2 + 4 y + 4 &\le y^5 + y^4 + 4 y^3 + 4 y^2 + 4 y + 4 \\
(y + 2)^2 &\le (y + 1) (y^2 + 2)^2
\end{align*}
Set $x = y + 1$ (so we have $x \ge 1$).
\begin{align*}
(x + 1)^2 &\le x ( (x - 1)^2 + 2)^2 \\
x &\le {x^2 ( (x - 1)^2 + 2)^2 \over (x + 1)^2} \\
\sqrt{x} &\le {x ((x - 1)^2 + 2) \over x + 1} \\
{x + 1 \over x} \sqrt{x} &\le (x - 1)^2 + 2 \\
{1 \over 2}
\left(
\sqrt{x} + {1 \over \sqrt{x}}
\right) &\le
1 + {1 \over 2} (x - 1)^2
\end{align*}
Thus, because $x_{n+1} = (\sqrt{x_n} + 1 /\sqrt{x_n})/2$, we have
\[
x_{n+1} - 1 \le {1 \over 2} (x_n - 1)^2 .
\]
From this equation, we may derive the bound
\[
x_n -1 \le \frac{1}{2^{2^n-1}} (x_0 - 1)^{2^n} .
\]
This is a much better bound!  It approaches zero far more rapidly
than any exponential function, so we have superlinear convergence.
%%%%%
%%%%%
\end{document}
