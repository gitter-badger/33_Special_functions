\documentclass[12pt]{article}
\usepackage{pmmeta}
\pmcanonicalname{ComplexArithmeticgeometricMean}
\pmcreated{2013-03-22 17:10:05}
\pmmodified{2013-03-22 17:10:05}
\pmowner{rspuzio}{6075}
\pmmodifier{rspuzio}{6075}
\pmtitle{complex arithmetic-geometric mean}
\pmrecord{15}{39480}
\pmprivacy{1}
\pmauthor{rspuzio}{6075}
\pmtype{Result}
\pmcomment{trigger rebuild}
\pmclassification{msc}{33E05}
\pmclassification{msc}{26E60}

\endmetadata

% this is the default PlanetMath preamble.  as your knowledge
% of TeX increases, you will probably want to edit this, but
% it should be fine as is for beginners.

% almost certainly you want these
\usepackage{amssymb}
\usepackage{amsmath}
\usepackage{amsfonts}

% used for TeXing text within eps files
%\usepackage{psfrag}
% need this for including graphics (\includegraphics)
%\usepackage{graphicx}
% for neatly defining theorems and propositions
%\usepackage{amsthm}
% making logically defined graphics
%%\usepackage{xypic}

% there are many more packages, add them here as you need them

% define commands here

\begin{document}
It is also possible to define the arithmetic-geometric mean for
complex numbers.  To do this, we first must make the geometric
mean unambiguous by choosing a branch of the square root.  We
may do this as follows: Let $a$ and $b$ br two non-zero complex 
numbers such that $a \neq sb$ for any real number $s < 0$.  Then
we will say that $c$ is the geometric mean of $a$ and $b$ if
$c^2 = ab$ and $c$ is a convex combination of $a$ and $b$ (i.e.
$c = s a + t b$ for positive real numbers $s$ and $t$). 

Geometrically, this may be understood as follows:  The condition
$a \neq sb$ means that the angle between $0a$ and $0b$ differs
from $\pi$.  The square root of $ab$ will lie on a line bisecting
this angle, at a distance $\sqrt{|ab|}$ from $0$.  Our condition
states that we should choose $c$ such that $0c$ bisects the angle
smaller than $\pi$, as in the figure below:

\[
\begin{xy}
,(2,-1)*{0}
,(0,0)
;(50,50)**@{-}
;(52,52)*{b}
,(0,0)
;(-16,16)**@{-}
,(-18,18)*{a}
,(0,0)
;(0,40)**@{-}
,(0,42)*{c}
,(0,0)
;(0,-40)**@{-}
,(0,-42)*{-c}
\end{xy}
\]

Analytically, if we pick a polar representation $a = |a| e^{i \alpha}$,
$b = |b| e^{i \beta}$ with $|\alpha - \beta| < \pi$, then $c = \sqrt{|ab|}
e^{i {\alpha + \beta \over 2}}$.  Having clarified this preliminary item,
we now proceed to the main definition.

As in the real case, we will define sequences of geometric and arithmetic
means recursively and show that they converge to the same limit.  With our
convention, these are defined as follows:
\begin{align*}
g_0 &= a \\
a_0 &= b \\
g_{n+1} &= \sqrt{a_n g_n} \\
a_{n+1} &= {a_n + g_n \over 2}
\end{align*}

We shall first show that the phases of these sequences converge.  As above,
let us define $\alpha$ and $\beta$ by the conditions $a = |a| e^{i \alpha}$,
$b = |b| e^{i \beta}$, and $|\alpha - \beta| < \pi$.  Suppose that $z$ and
$w$ are any two complex numbers such that $z = |z| e^{i \theta}$ and $w = 
|w| e^{i \phi}$ with $|\phi - \theta| < \pi$.  Then we have the following:
\begin{itemize}
\item  The phase of the geometric mean of $z$ and $w$ can be chosen to lie
between $\theta$ and $\phi$.  This is because, as described earlier, this
phase can be chosen as $(\theta + \phi)/2$.
\item The phase of the arithmetic mean of $z$ and $w$ can be chosen to lie
between $\theta$ and $\phi$.
\end{itemize}
By a simple induction argument, these two facts imply that we can introduce
polar representations $a_n = |a_n| e^{i \theta_n}$ and $g_n = |g_n|
e^{i \phi_n}$ where, for every $n$, we find that $\theta_n$ lies between
$\alpha$ and $\beta$ and likewise $\phi_n$ lies between $\alpha$ and $\beta$.
Furthermore, since $\phi_{n+1} = (\phi_n + \theta_n) / 2$ and $\theta_{n+1}$ 
lies between $\phi_n$ and $\theta_n$, it follows that
\[
|\phi_{n+1} - \theta_{n+1}| \le
{1 \over 2} |\phi_n - \theta_n| .
\]
Hence, we conclude that $|\phi_n - \theta_n| \to 0$ as $n \to \infty$.  By 
the principle of nested intervals, we further conclude that the sequences
$\{\theta_n\}_{n=0}^\infty$ and $\{\phi_n\}_{n=0}^\infty$ are both convergent 
and converge to the same limit.

Having shown that the phases converge, we now turn our attention to the
moduli.  Define $m_n = \max (|a_n|, |g_n|)$.  Given any two complex
numbers $z,w$, we have
\[
|\sqrt{zw}| \le \max (|z|,|w|)
\]
and
\[
\left| {z + w \over 2} \right| \le
\max (|z|,|w|) ,
\]
so this sequence $\{m_n\}_{n=0}^\infty$ is decreasing.  Since it bounded from
below by $0$, it converges.

Finally, we consider the ratios of the moduli of the arithmetic and geometric 
means.  Define $x_n = |a_n| / |g_n|$.  As in the real case, we shall derive a
recursion relation for this quantity:
\begin{align*}
x_{n+1} &= 
{|a_{n+1}| \over |g_{n+1}|} \\ &= 
{|a_n + g_n| \over 2 \sqrt{|a_n g_n|}} \\&=
{\sqrt{|a_n^2| + 2 |a_n| |g_n| \cos (\theta_n - \phi_n) + |g_n|^2} 
\over 2 \sqrt{|a_n g_n|}} \\&=
{1 \over 2}
\sqrt{ {|a_n| \over |g_n|} +
2 \cos (\theta_n - \phi_n) +
{|g_n| \over |a_n|}} \\ &=
{1 \over 2}
\sqrt {x_n + 2 \cos (\theta_n - \phi_n) + {1 \over x_n}}
\end{align*}

For any real number $x \ge 1$, we have the following:
\begin{align*}
x - 1 &\ge 0 \\
(x - 1)^2 &\ge 0 \\
x^2 - 2 x + 1 &\ge 0 \\
x^2 + 1 &\ge 2x \\
x + {1 \over x} &\ge 2
\end{align*}
If $0 < x < 1$, then $1/x > 1$, so we can swithch the roles of $x$ and $1/x$ and
conclude that, for all real $x > 0$, we have
\[
x + {1 \over x} \ge 2 .
\]
Applying this to the recursion we just derived and making use of the half-angle
identity for the cosine, we see that
\[
x_{n+1} \ge
{1 \over 2}
\sqrt{2 + 2 \cos (\theta_n - \phi_n)} =
\cos \left( {\theta_n - \phi_n \over 2} \right) . 
\]
%%%%%
%%%%%
\end{document}
