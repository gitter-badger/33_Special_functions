\documentclass[12pt]{article}
\usepackage{pmmeta}
\pmcanonicalname{ProductRepresentationsOfJacobivarthetaFunctions}
\pmcreated{2013-03-22 14:52:13}
\pmmodified{2013-03-22 14:52:13}
\pmowner{rspuzio}{6075}
\pmmodifier{rspuzio}{6075}
\pmtitle{product representations of Jacobi $\vartheta$ functions}
\pmrecord{4}{36546}
\pmprivacy{1}
\pmauthor{rspuzio}{6075}
\pmtype{Theorem}
\pmcomment{trigger rebuild}
\pmclassification{msc}{33E05}

\endmetadata

% this is the default PlanetMath preamble.  as your knowledge
% of TeX increases, you will probably want to edit this, but
% it should be fine as is for beginners.

% almost certainly you want these
\usepackage{amssymb}
\usepackage{amsmath}
\usepackage{amsfonts}

% used for TeXing text within eps files
%\usepackage{psfrag}
% need this for including graphics (\includegraphics)
%\usepackage{graphicx}
% for neatly defining theorems and propositions
%\usepackage{amsthm}
% making logically defined graphics
%%%\usepackage{xypic}

% there are many more packages, add them here as you need them

% define commands here
\begin{document}
The Jacobi theta functions can be expressed as infinite products:

$$\vartheta_1 (z;q) = 2 q^{1/4} \sin z \prod_{n=1}^\infty (1 - q^{2n}) (1 - 2 q^{2n} \cos 2 z + q^{4n})$$
$$\vartheta_2 (z;q) = 2 q^{1/4} \cos z \prod_{n=1}^\infty (1 - q^{2n}) (1 + 2 q^{2n} \cos 2 z + q^{4n})$$
$$\vartheta_3 (z;q) = \prod_{n=1}^\infty (1 - q^{2n}) (1 + 2 q^{2n-1} \cos 2 z + q^{4n-2})$$
$$\vartheta_4 (z;q) = \prod_{n=1}^\infty (1 - q^{2n}) (1 - 2 q^{2n-1} \cos 2 z + q^{4n-2})$$
%%%%%
%%%%%
\end{document}
