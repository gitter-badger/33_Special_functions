\documentclass[12pt]{article}
\usepackage{pmmeta}
\pmcanonicalname{ProofOfMultiplicationFormulaForGammaFunction}
\pmcreated{2013-03-22 14:44:10}
\pmmodified{2013-03-22 14:44:10}
\pmowner{rspuzio}{6075}
\pmmodifier{rspuzio}{6075}
\pmtitle{proof of multiplication formula for gamma function}
\pmrecord{9}{36369}
\pmprivacy{1}
\pmauthor{rspuzio}{6075}
\pmtype{Proof}
\pmcomment{trigger rebuild}
\pmclassification{msc}{33B15}
\pmclassification{msc}{30D30}

\endmetadata

% this is the default PlanetMath preamble.  as your knowledge
% of TeX increases, you will probably want to edit this, but
% it should be fine as is for beginners.

% almost certainly you want these
\usepackage{amssymb}
\usepackage{amsmath}
\usepackage{amsfonts}

% used for TeXing text within eps files
%\usepackage{psfrag}
% need this for including graphics (\includegraphics)
%\usepackage{graphicx}
% for neatly defining theorems and propositions
%\usepackage{amsthm}
% making logically defined graphics
%%%\usepackage{xypic}

% there are many more packages, add them here as you need them

% define commands here
\begin{document}
Define the function $f$ as
 $$f(z) = {n^{nz} \prod\limits_{k=0}^{n-1}
 \Gamma \left( z + {k \over n} \right) \over \Gamma (nz)}$$

By the functional equation of the gamma function,
 $$f(z+1) = {n^n n^{nz} \left( \prod\limits_{m=0}^{n-1}
 \Gamma \left (z + {m \over n} \right) \right) \prod\limits_{k=0}^{n-1}
 \left( z + {k \over n} \right) \over \Gamma (nz) \prod_{k=0}^{n-1} (nz + k) } = f(z)$$

Hence $f$ is a periodic function of $z$.  However, for large values
of $z$, we can apply the Stirling approximation formula to conclude
 $$f(z) = {(2 \pi)^{n/2} n^{nz} \prod\limits_{k=0}^{n-1} \left[ e^{-z-k/n}
  (z + k/n)^{z + k/n - 1/2} + O(e^{-z} (z+ k/n)^{z + k/n - 3/2}) \right] \over (2 \pi)^{1/2}
   e^{-nz} (nz)^{nz - 1/2} + O(e^{-nz} (nz)^{nz - 3/2}) } = $$
$$(2 \pi)^{(n-1)/2} n^{1/2} \, {\prod\limits_{k=0}^{n-1} \left[ e^{-k/n}
  (z + k/n)^{z + k/n - 1/2} + O((z+ k/n)^{z + k/n - 3/2}) \right] \over
  z^{nz - 1/2} + O(z^{nz - 3/2}) } = $$
$$(2 \pi)^{(n-1)/2} n^{1/2} \, {z^{1/2} \prod\limits_{k=0}^{n-1} \left[ e^{-k/n}
  \left( 1 + {k \over nz} \right)^{z + k/n - 1/2} z^{k/n - 1/2} + O((z+ k/n)^{k/n - 3/2}) \right] \over
  1 + O(z^{- 1}) }$$

Note that
 $$\prod\limits_{k=0}^{n-1} e^{-k/n} = e^{- \sum_{k=0}^{n-1}
k/n} = e^{(1-n)/2}$$
 $$z^{1/2} \prod\limits_{k=0}^{n-1} z^{k/n - 1/2} = z^{1/2} z^{\sum_{k=0}^{n-1}
(k/n - 1/2)} = z^{1/2 + (n-1)/2 - n/2} = 1$$

Also,
 $$\left( 1 + {k \over nz} \right)^z = e^{k/n} + O(z^{-1})$$
Hence, $f(z) = (2 \pi)^{(n-1)/2} n^{1/2} + O(z^{- 1})$.  Now, the only way for a function to be periodic and have a definite limit is for that function to be constant.  Therefore, $f(z) = (2 \pi)^{(n-1)/2} n^{1/2}$.  Writing out the definition of $f$ and rearranging gives the multiplication formula.
%%%%%
%%%%%
\end{document}
